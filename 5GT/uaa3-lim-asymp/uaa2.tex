% Created 2024-11-25 Mon 19:23
% Intended LaTeX compiler: pdflatex
\documentclass[a4paper,12pt]{report}
\usepackage[utf8]{inputenc}
\usepackage[T1]{fontenc}
\usepackage{graphicx}
\usepackage{longtable}
\usepackage{wrapfig}
\usepackage{rotating}
\usepackage[normalem]{ulem}
\usepackage{amsmath}
\usepackage{amssymb}
\usepackage{capt-of}
\usepackage{hyperref}
\usepackage{my-conf}
\renewcommand{\arraystretch}{1.2}
\newcommand{\IR}{\mathbb{R}}
\newcommand{\IZ}{\mathbb{Z}}
\newcommand{\IQ}{\mathbb{Q}}

\newcommand{\IN}{\mathbb{N}}
\renewcommand{\emptyset}{\varnothing}
\newcommand{\dom}{\mathrm{dom}}
\newcommand{\im}{\mathrm{im}}
\newcommand{\jdot}[1]{ \makebox[#1]{\dotfill}}
\newcommand{\tog}{\stackrel[<]{}{\to}}
\newcommand{\tod}{\stackrel[>]{}{\to}}
\newcommand{\pinf}{+\infty}
\newcommand{\minf}{-\infty}
\newcommand{\pminf}{\pm\infty}
\newcommand{\mpinf}{\mp\infty}
\newcommand{\FI}{\textbf{F.I.}}
\newcommand{\abs}[1]{|#1|}
\newcommand{\rad}{\text{rad}}
\author{Quentin Lambotte}
\date{2024-11-25}
\title{Les suites\\\medskip
\large 5UAA2}
\hypersetup{
 pdfauthor={Quentin Lambotte},
 pdftitle={Les suites},
 pdfkeywords={},
 pdfsubject={notes de cours de l'uaa2, 5GTT, math 4p},
 pdfcreator={Emacs 29.3 (Org mode 9.6.15)}, 
 pdflang={French}}
\begin{document}

\begin{titlepage}
\begin{center}
\vspace{2cm}
\Huge
\textbf{5UAA3: Limites et asymptotes}
\vspace{2cm}

\end{center}

\vspace{1cm}
\singlespacing
\small
\uline{Compétences à développer dans ce chapitre :}
\begin{itemize}
\item Connaître
\begin{itemize}
\item Identifier dans l’expression analytique d’une fonction donnée les
fonctions usuelles, les opérations et leur hiérarchie.
\item Donner un exemple de limite de fonction illustrant un cas d'indétermination.
\end{itemize}
\item Appliquer
\begin{itemize}
\item Déterminer, à partir de l’expression analytique d’une fonction,
son domaine et les limites qui apportent des informations sur son
graphique.
\item Calculer des limites et les interpréter graphiquement.
\item Apparier des graphiques et des informations sur les limites et les
asymptotes d’une fonction.
\item Traduire en termes de limites les comportements asymptotiques
d’une fonction, à partir de son graphique.
\item Rechercher les équations des asymptotes au graphique d’une
fonction.
\item Utiliser le comportement asymptotique d’une fonction pour
approcher sa valeur en un point.
\end{itemize}
\item Transférer
\begin{itemize}
\item Esquisser le graphique d’une fonction vérifiant certaines
conditions sur les limites et les asymptotes.
\item Rechercher l’expression analytique d’une fonction répondant à
certaines conditions relatives à ses limites et à ses asymptotes.
\end{itemize}
\end{itemize}
\end{titlepage}
\onehalfspacing
\chapter{Généralités sur les fonctions}
\label{sec:org7b2e85c}
\section{Rappels}
\label{sec:orgcd02b42}
Voir feuille a3 en annexe.
\section{Asymptotes}
\label{sec:orga6f44bf}
Lors de votre analyse des fonctions de référence en 4e, vous avez rencontré la
fonction \(f(x)=\dfrac{1}{x}\). Cette fonction présente une particularité:
\begin{center}
\includestandalone[width=0.6\linewidth]{figures/fig00}
\end{center}

La particularité est que cette fonction admet deux asymptotes: l'une est
horizontale et l'autre est verticale.

Intuitivement, une asymptote au graphique d'une fonction est une droite qui
semble se rapprocher de plus en plus du graphique de cette fonction.

Un des objectifs de cette uaa est de fournir des outils qui nous permettrons de
détecter ces asymptotes.

Il existe trois type d'asymptotes: verticales, horizontale et oblique.

Voici quelques exemples:
\begin{center}
\includestandalone[width=\linewidth]{figures/fig35}
\end{center}

\begin{exercice}
Pour les fonctions représentées ci-dessus,

\begin{enumerate}
\item donne la nature de chaque asymptote: verticale, horizontale ou oblique;
\item donne ensuite une équation cartésienne de chaque asymptote.
\end{enumerate}
\end{exercice}

\begin{exercice}
Dans le repère ci-dessous, invente une fonction qui a une asymptote verticale
d'équation \(x=4\).
\begin{center}
\includestandalone[width=0.7\linewidth]{figures/fig18}
\end{center}
\end{exercice}

\begin{exercice}
Dans le repère ci-dessous, invente une fonction qui a une asymptote horizontale
d'équation \(y=-2\).
\begin{center}
\includestandalone[width=0.7\linewidth]{figures/fig18}
\end{center}
\end{exercice}

\begin{exercice}
Dans le repère ci-dessous, invente une fonction qui a une asymptote oblique
d'équation \(y=\dfrac{x}{2}-1\).
\begin{center}
\includestandalone[width=0.7\linewidth]{figures/fig18}
\end{center}
\end{exercice}
\newpage

\section{Recherche algébrique du domaine d'une fonction}
\label{sec:org74b675c}
Jusqu'à présent, nous nous sommes contentés d’observer le domaine d'une fonction
à partir de son graphique. Nous allons dorénavant déterminer celui-ci à partir de son
expression analytique.
Pour ce faire, on commence par poser des conditions d'existence. Dans ce cours,
nous rencontrerons les conditions d'existence suivantes:
\begin{center}
  \renewcommand{\arraystretch}{3}
  \begin{tabular}{|l|l|}\hline
    Expression& Condition d'existence\\\hline
    $\dfrac{1}{a}$&a\neq 0\\\hline
    $\sqrt{a}$&$a\ge 0$\\\hline
    $\dfrac{1}{\sqrt{a}}$&$a>0$\\\hline
  \end{tabular}
\end{center}
\begin{exercice}
Pour les expressions suivantes, indiques quel est le type de condition d'existence adéquat:

\begin{enumerate}
\item \(f(x)=\dfrac{1}{x-4}\)
\begin{center}
\square \(a\neq 0\) \hspace{1cm}\square \(a\ge 0\) \hspace{1cm}\square \(a>0\)
\hspace{1cm} \square aucune
\end{center}

\item \(f(x)=\dfrac{x}{\sqrt{1+x}}\)
\begin{center}
\square \(a\neq 0\) \hspace{1cm}\square \(a\ge 0\) \hspace{1cm}\square \(a>0\)
\hspace{1cm} \square aucune
\end{center}
\item \(f(x)=\sqrt{x+1}\)
\begin{center}
\square \(a\neq 0\) \hspace{1cm}\square \(a\ge 0\) \hspace{1cm}\square \(a>0\)
\hspace{1cm} \square aucune
\end{center}
\item \(f(x)=x-5\)
\begin{center}
\square \(a\neq 0\) \hspace{1cm}\square \(a\ge 0\) \hspace{1cm}\square \(a>0\)
\hspace{1cm} \square aucune
\end{center}
\end{enumerate}
\end{exercice}
\begin{exercice}
Voici l'expression analytique de huit fonctions. Calcule le domaine de
définition de chacune de ces fonctions.
\par \setlength{\columnseprule}{0 pt}
          \begin{minipage}[t]{\linewidth}
          \begin{multicols}{3}

\begin{enumerate}
\item \(f(x)=2x^2-x^4\)

\item \(g(x)=\dfrac{x^2-2}{x-2}\)

\item \(h(x)=\sqrt{x-2}\)

\item \(i(x)=\dfrac{x+1}{x-2}\)

\item \(j(x)=\dfrac{1}{\sqrt{x-2}}\)

\item \(k(x)=\dfrac{\sqrt{x+2}}{x^2-1}\)

\item \(l(x)=\dfrac{x^3+2}{x^2-4}\)

\item \(m(x)=\dfrac{x^3+2}{(x-2)^2}\)
\end{enumerate}


\end{multicols}\end{minipage}
\end{exercice}
\chapter{Opérations sur les fonctions}
\label{sec:org703fb48}
\section{Somme, différence, produit et quotient}
\label{sec:orgda1976b}
Les opérations usuelles sur les nombres peuvent être prolongées en des
opérations sur les fonctions.
\begin{definition}
Soit \(f\) et \(g\) deux fonctions. Alors:

\begin{itemize}
\item la fonction \(f+g\) est définie par l'équation \((f+g)(x)=f(x)+g(x)\)

\item la fonction \(f-g\) est définie par l'équation \((f-g)(x)=f(x)-g(x)\)

\item la fonction \(fg\) est définie par l'équation \((fg)(x)=f(x)\cdot g(x)\)

\item la fonction \(\dfrac{f}{g}\) est définie par l'équation \(\left(\dfrac{f}{g}\right)(x)=\dfrac{f(x)}{g(x)}\)
\end{itemize}
\end{definition}
Ainsi, pour calculer l'image d'un nombre pour la somme de deux fonctions, il
faut calculer la somme des images.

\begin{exemple}
Considérons la fonctions \(f(x)=x\) et la fonction \(g(x)=\dfrac{1}{x}\). Voici le
graphique de ces deux fonctions.
\begin{center}
\includestandalone[width=0.55\linewidth]{figures/fig02}
\end{center}
\begin{enumerate}
\item Complète le tableau suivant
\begin{center}
\begin{tabular}{|l|l|l|l|l|l|l|l|}
\hline
$x$             & $f(x)$          & $g(x)$ & $f(x)+g(x)$ & $f(x)+g(x)$ & $f(x)-g(x)$ & $f(x)g(x)$ & $\dfrac{f(x)}{g(x)}$ \\ \hline
-4              & -4              &        &             &             &             &            &                      \\ \hline
-3              & -3              &        &             &             &             &            &                      \\ \hline
-1              & -1              &        &             &             &             &            &                      \\ \hline
$-\dfrac{1}{4}$ & $-\dfrac{1}{4}$ &        &             &             &             &            &                      \\ \hline
$-\dfrac{1}{2}$ & $-\dfrac{1}{2}$ &        &             &             &             &            &                      \\ \hline
0               & 0               &        &             &             &             &            &                      \\ \hline
$\dfrac{1}{2}$  & $\dfrac{1}{2}$  &        &             &             &             &            &                      \\ \hline
$\dfrac{1}{4}$  & $\dfrac{1}{4}$  &        &             &             &             &            &                      \\ \hline
1               & 1               &        &             &             &             &            &                      \\ \hline
2               & 2               &        &             &             &             &            &                      \\ \hline
3               & 3               &        &             &             &             &            &                      \\ \hline
4               & 4               &        &             &             &             &            &                      \\ \hline
\end{tabular}
\end{center}

\item Donne le domaine de chacune des fonctions \(f+g\), \(f-g\), \(fg\) et
\(\dfrac{f}{g}\).
\vspace{2cm}

\item Trace le graphique de \(f+g\) dans le repère précédent.

\item Que constates-tu?
\vspace{2cm}
\end{enumerate}
\end{exemple}

\begin{exercice}
Soit \(f\) et \(g\) deux fonctions. Donne le domaine de définition des fonctions \(f+g\), \(f-g\), \(fg\) et
   \(\dfrac{f}{g}\).
\end{exercice}
\newpage
\begin{exercice}
Soit les fonctions \(h(x)=x^2-1\) et \(t(x)=\sqrt{x+1}\). Complète les égalités
suivantes:
\par \setlength{\columnseprule}{0 pt}
          \begin{minipage}[t]{\linewidth}
          \begin{multicols}{2}

\includestandalone[width=\linewidth]{figures/fig01}

\begin{enumerate}
\item \(\dom(h)=\) \dotfill
\item \(\dom(t)=\) \dotfill
\item \(\dom(h+t)=\) \dotfill
\item \(\dom(h-t)=\) \dotfill
\item \(\dom(ht)=\) \dotfill
\item \(\dom(h/t)=\) \dotfill
\item \(ht(x)=\) \dotfill
\item \(ht(0)=\) \dotfill
\item \(ht(-4)=\) \dotfill
\item \(ht(-1)=\) \dotfill
\item \((h-t)(0)=\) \dotfill
\item \((h-t)(-4)=\) \dotfill
\item \((h-t)(-1)=\) \dotfill
\end{enumerate}


\end{multicols}\end{minipage}
\end{exercice}

\section{Composition}
\label{sec:org34146eb}
\subsection{Exemples}
\label{sec:org8099cda}
\begin{exemple}
Soit \(h(x)=\sqrt{x+1}\). Calcule les images suivantes:

\begin{itemize}
\item \(h(0)=\) \dotfill
\item \(h(101)=\) \dotfill
\item \(h(-1)=\) \dotfill
\end{itemize}

Quelles sont les différentes étapes que tu as appliquées pour calculer ces
images?

\vspace{2cm}

La fonction \(h\) est construites à partir de deux autres fonctions. Peux-tu les
identifier?

\vspace{2cm}
\end{exemple}
\newpage
\begin{exemple}
Soit \(j(x)=x^2+1\). Calcule les images suivantes:

\begin{itemize}
\item \(j(0)=\) \dotfill
\item \(j(101)=\) \dotfill
\item \(j(-1)=\) \dotfill
\end{itemize}

Quelles sont les différentes étapes que tu as appliquées pour calculer ces
images?

\vspace{2cm}

La fonction \(j\) est construites à partir de deux autres fonctions. Peux-tu les
identifier?
\vspace{2cm}
\end{exemple}
\subsection{Composée de deux fonctions}
\label{sec:org2120824}

\begin{definition}
Soit \(f\) et \(g\) deux fonctions. La fonction composée de \(f\) après \(g\), notée
\(f\circ g\), est définie par l'équation
\[
(f\circ g)(x)=f(g(x))
\]
\end{definition}

On peut schématiser cette définition de la manière suivante:

\vspace{3cm}

\begin{exemple}
Soit \(f(x)=\dfrac{1}{x}\) et \(f(x)=x-2\).
\par \setlength{\columnseprule}{0 pt}
          \begin{minipage}[t]{\linewidth}
          \begin{multicols}{2}

Analysons la fonction \(f\circ g\):

\begin{enumerate}
\item Expression analytique:
\end{enumerate}
\vspace{1cm}

\begin{enumerate}
\item Domaine
\end{enumerate}


Analysons la fonction \(g\circ f\):

\begin{enumerate}
\item Expression analytique:
\end{enumerate}
\vspace{1cm}

\begin{enumerate}
\item Domaine
\end{enumerate}


\end{multicols}\end{minipage}
\vspace{2cm}
\end{exemple}

De cet exemple, on peut retenir une remarque importante:
\begin{remarque}
La composition de deux fonctions n'est pas commutative!!! Cela veut dire qu'en
général les fonctions \(f\circ g\) et \(g\circ f\) sont différentes!
\end{remarque}

Comment déterminer le domaine d'une fonction composée? Soit \(f\) et \(g\) deux
fonctions. Commençons par un schéma:
\vspace{10cm}

Peux-tu décrire en français comment le domaine de \(f\circ g\) s'obtient à partir
du domaine de \(f\) et du domaine de \(g\)?
\vspace{3cm}

\begin{propriete}
Soit \(f\) et \(g\) deux fonctions. Alors le domaine de \(f\circ g\) est donné par
l'égalité suivante:
\vspace{1cm}
\end{propriete}

\begin{exercice}
Dans les cas suivants, donne une expression analytique de \(f\circ g\) puis de
\(g\circ f\).

\par \setlength{\columnseprule}{0 pt}
          \begin{minipage}[t]{\linewidth}
          \begin{multicols}{2}
\begin{enumerate}
\item \(f(x)=x^2-1\) et \(g(x)=3x+5\)

\item \(f(x)=\sqrt{x}\) et \(g(x)=x^2-1\)

\item \(f(x)=\dfrac{2}{x}\) et \(g(x)=x^2-1\)

\item \(f(x)=\dfrac{1}{x-1}\) et \(g(x)=x+1\)
\end{enumerate}


\end{multicols}\end{minipage}
\end{exercice}

\begin{exercice}
Soit \(h(x)=\dfrac{1}{x-3}\) et \(i(x)=x^2+3\). Donne l'exprresion analytique de \(h\circ i\) et \(i\circ
h\). Donne ensuite le domaine de ces deux fonctions composées.
\end{exercice}

\begin{exercice}
Voici le graphique de deux fonctions \(f\) et \(g\). Voici leur graphique:
\begin{center}
\includestandalone[width=0.6\linewidth]{figures/fig03}
\end{center}
\par \setlength{\columnseprule}{0 pt}
          \begin{minipage}[t]{\linewidth}
          \begin{multicols}{2}
\begin{itemize}
\item \(g(f(2))=\) \dotfill

\item \((f\circ g)(-2)=\) \dotfill

\item \(f(g(0))=\) \dotfill

\item \(g(f(0))=\) \dotfill

\item \((g\circ f)(-3)=\) \dotfill

\item \((f+g)(3)=\) \dotfill

\item \((f-g)(0)=\) \dotfill

\item \((fg)(-4)=\) \dotfill

\item \((fg)(-5)=\) \dotfill

\item \((f/g)(3)=\) \dotfill
\end{itemize}


\end{multicols}\end{minipage}
\end{exercice}


\section{Décomposition d'une fonction: exercices}
\label{sec:orgf5b9e71}
Décomposer une fonction, c'est décrire comment la construire à partir des
opérations et de fonctions simples. Dans ce cours, les opérations seront
\[
+,-,\cdot,/,\circ
\]
et les fonctions simples seront les fonctions de référence:
\[
x,x^2,x^3,x^n,\dfrac{1}{x}, \sqrt{x},\sqrt[3]{x},|x|,k.
\]

Voici deux exemples:
\newpage
\begin{exemple}
Soit \(f(x)=\sqrt{x^3+\dfrac{1}{x}}\). Décomposons \(f\).
\vspace{10cm}
\end{exemple}

\begin{exemple}
Soit \(g(x)=|x+1|-\sqrt{x^2+1}\). Décomposons \(g\).
\vspace{10cm}
\end{exemple}

\begin{exercice}
Décompose les fonctions suivantes:
\par \setlength{\columnseprule}{0 pt}
          \begin{minipage}[t]{\linewidth}
          \begin{multicols}{2}
\begin{enumerate}
\item \(h(x)=\dfrac{1}{x-3}\)

\item \(p(x)=\dfrac{1}{x^2}\)

\item \(r(x)=\sqrt{1+x^2}+x\)

\item \(u(x)=\left(\dfrac{1}{x}+4\right)^2\)
\end{enumerate}


\end{multicols}\end{minipage}
\end{exercice}
\chapter{Limites d'une fonction}
\label{sec:org5091752}

\section{Introduction}
\label{sec:orga52d9b9}
Voici le graphe d'une fonction \(f\) un peu alambiquée:
\begin{center}
\includestandalone[width=\linewidth]{figures/fig1}
\end{center}

En 3e et en 4e, vous avez vu divers concepts pour décrire la fonction
à partir de son graphique.
\par \setlength{\columnseprule}{0 pt}
          \begin{minipage}[t]{\linewidth}
          \begin{multicols}{2}
\begin{enumerate}
\item \(\dom(f)=\dotfill\)
\item \(\im(f)=\dotfill\)
\item o.à.o = \dotfill
\item Racines:\dotfill
\end{enumerate}


\end{multicols}\end{minipage}
Vous avez aussi construit deux tableaux qui décrivent les changements
de signe et les variations:


\begin{center}
\includestandalone[width=\linewidth]{figures/tab1}
\end{center}
\begin{center}
\includestandalone[width=\linewidth]{figures/tab2}
\end{center}

Nous allons utiliser le concept de limite afin de préciser les
informations données dans ces deux tableaux. Nous allons répondre à
divers questions qui vont mettre en évidence le comportement étrange
de \(f\) à certains points de son domaine, aux endroits où \(f\) n'est pas
définie et quand la variable indéquendante prend des valeurs très
grandes ou très petites.

\begin{enumerate}
\item Que peux-tu dire de \(f(x)\) quand \(x\) est proche de \(-9\)? (On dira
aussi: quand \(x\) est au voisinage de \(-9\)).

\dotfill

\dotfill

Notation: \dotfill

Observation graphique: \dotfill
\item Que peux-tu dire de \(f(x)\) quand \(x\) est proche de \(-6\)?

\dotfill

\dotfill

Notation: \dotfill

Observation graphique: \dotfill
\item Que peux-tu dire de \(f(x)\) quand \(x\) est proche de \(-2\)?

\dotfill

\dotfill

Notation: \dotfill

Observation graphique: \dotfill
\item Que peux-tu dire de \(f(x)\) quand \(x\) est proche de \(1\)?

\dotfill

\dotfill

Notation: \dotfill

Observation graphique: \dotfill
\item Que peux-tu dire de \(f(x)\) quand \(x\) est proche de \(6\)?

\dotfill

\dotfill

Notation: \dotfill

Observation graphique: \dotfill
\item Que peux-tu dire de \(f(x)\) quand \(x\) est très petit? (On dira
aussi proche de \(-\infty\))

\dotfill

\dotfill

Notation:  \dotfill

Observation graphique: \dotfill
\item Que peux-tu dire de \(f(x)\) quand \(x\) est très grand? (On dira
aussi proche de \(+\infty\))

\dotfill

\dotfill

Notation: \dotfill

Observation graphique: \dotfill
\end{enumerate}

La notion de limite est un outil essentiel pour comprendre le
comportement des fonctions. Cette notion permet de mettre en évidence
le comportement d'une fonction aux bornes de son domaine, en des
valeurs interdites ou bien en des points du domaine où la fonction a
une brisure.

Voici un exercice sur les fonctions de référence rencontrées en
4e.

\begin{exercice}
Voici les graphiques des fonctions de référence. Complète ensuite le
tableau.

\begin{center}
\includestandalone[width=\linewidth]{figures/fig2}
\end{center}

\begin{center}
\begin{tabular}{l|l|l|l|l|l|l|}
\cline{2-7}
                                          & $\lim\limits_{x\to -2}f(x)$ & $\lim\limits_{x\to 0}f(x)$ & $\lim\limits_{x\to 1}f(x)$ & $\lim\limits_{x\to a}f(x),\,\,  a\in\dom(f)$ & $\lim\limits_{x\to \minf}f(x)$ & $\lim\limits_{x\to \pinf}f(x)$ \\ \hline
\multicolumn{1}{|l|}{$f(x)=k$}            &                             &                            &                            &                                            &                                &                                \\ \hline
\multicolumn{1}{|l|}{$f(x)=x$}            &                             &                            &                            &                                            &                                &                                \\ \hline
\multicolumn{1}{|l|}{\rule[-12pt]{0pt}{33pt}$f(x)=\dfrac{1}{x}$} &                             &                            &                            &                                            &                                &                                \\ \hline
\multicolumn{1}{|l|}{$f(x)=|x|$}          &                             &                            &                            &                                            &                                &                                \\ \hline
\multicolumn{1}{|l|}{$f(x)=x^2$}          &                             &                            &                            &                                            &                                &                                \\ \hline
\multicolumn{1}{|l|}{$f(x)=\sqrt{x}$}     &                             &                            &                            &                                            &                                &                                \\ \hline
\multicolumn{1}{|l|}{$f(x)=x^3$}          &                             &                            &                            &                                            &                                &                                \\ \hline
\multicolumn{1}{|l|}{$f(x)=\sqrt[3]{x}$}  &                             &                            &                            &                                            &                                &                                \\ \hline
\end{tabular}
\end{center}
\end{exercice}
\section{Définition et interprétations graphiques}
\label{sec:org69eec05}
\begin{definition}
Soit \(f\) une fonction et \(a,b\in\IR\cup\{-\infty,+\infty\}\). On dit
que la limite de \(f(x)\) quand \(x\) tend vers \(a\) vaut \(b\) si: lorsque
\(x\) est proche \textbf{et} différent de \(a\), \(f(x)\) est proche de \(b\). Ceci se note:
\[
\lim\limits_{x\to a}f(x)=b.
\]
\end{definition}

\begin{remarque}
Dans la définition précédente:
\begin{itemize}
\item \(a\) et \(b\) peuvent être infinis.
\item \(x\) (ou \(f(x)\)) est proche de \(+\infty\) signifie:
\(x\) (ou \(f(x)\)) est très grand.
\item \(x\) (ou \(f(x)\)) est proche de \(-\infty\) signifie:
\(x\) (ou \(f(x)\)) est très petit.
\end{itemize}
\end{remarque}

\begin{exemple}
Soit \(f\) une fonction. L'expression \(\lim\limits_{x\to -10}f(x)=-\infty\)
signifie que  \(f(x)\) est proche de \(-\infty\) lorsque \(x\) est proche et
différent de \(-10\).

Graphiquement, cela se produit lorsque le graphe de \(f\) a une
asymptote verticale d'équation \(x=-10\). En particulier, \(-10\) n'est
pas dans le domaine de \(f\).

\begin{center}
\includestandalone[width=0.7\linewidth]{figures/fig10}
\end{center}
\end{exemple}


\begin{exercice}
Soit \(f\) une fonction. Que veut dire \(\lim\limits_{x\to -\infty}
f(x)=-2\)? Quelle interprétation graphique peux-tu donner de cette
égalité? Illustre avec un exemple.
\end{exercice}

\begin{exercice}
Voici des interpretations graphiques. A quel(s) cas de figure
correspondent-elles?

\begin{itemize}
\item Le graphique de \(f\) a une A.H. à gauche.
\item Le graphique de \(f\) a une A.H. à droite.
\item Le graphique de \(f\) a une A.V. d'équation \(x=a\).
\end{itemize}
\end{exercice}

\begin{exercice}
Voici des égalités. Quelles interprétations graphiques peux-tu donner?

\begin{itemize}
\item \(\lim\limits_{x\to\minf}g(x)=\pinf\)
\item \(\lim\limits_{x\to -11}h(x)=8\) et \(-11\notin \dom(h)\)
\item \(\lim\limits_{x\to 3}j(x)=0\), \(3\in\dom(j)\) et \(j(3)=4\).
\end{itemize}
\end{exercice}
\subsection{Limites à gauche et à droite}
\label{sec:org45889b0}
Nous avons vu dans l'exemple introductif qu'une limite peut ne pas
exister. Voici à nouveau ce graphique:

\begin{center}
\includestandalone[width=\linewidth]{figures/fig1}
\end{center}

Analysons les endroits problématiques:

\begin{enumerate}
\item au voisinage de \(-6\):\dotfill

\dotfill

\dotfill
\item au voisinage de \(-2\):\dotfill

\dotfill

\dotfill
\item au voisinage de \(6\):\dotfill

\dotfill

\dotfill
\end{enumerate}


\begin{definition}
Soit \(f\) une fonction, \(a\in\IR\) et \(b\in\IR\cup\{-\infty,+\infty\}\).
\begin{itemize}
\item On dit que la limite \textbf{à gauche} de \(f(x)\) quand \(x\) tend vers
\(a\) vaut \(b\) si: lorsque \(x\) est proche \textbf{et} strictement
inférieur à \(a\), \(f(x)\) est proche de \(b\).

Ceci se note: \(\lim\limits_{x\tog a}f(x)=b\).

\item On dit que la limite \textbf{à droite} de \(f(x)\) quand \(x\) tend vers
\(a\) vaut \(b\) si: lorsque \(x\) est proche \textbf{et} strictement
supérieur à \(a\), \(f(x)\) est proche de \(b\).

Ceci se note: \(\lim\limits_{x\tod a}f(x)=b\).
\end{itemize}
\end{definition}

L'intérêt de ces deux notions est la propriété suivante:

\begin{propriete}
Soit \(f\) une fonction, \(a\in\IR\) et
\(b\in\IR\cup\{-\infty,+\infty\}\). Alors

\[
\lim\limits_{x\to a}f(x)=b \Leftrightarrow \lim\limits_{x\tog a}f(x)=b \text{ et } \lim\limits_{x\tod a}f(x)=b.
\]
\end{propriete}

\begin{remarque}
 En pratique, cette propriété fournit une stratégie pour démontrer
qu'une limite n'existe pas: si les limites à gauche et à droite sont
différentes, alors la limite n'existe pas.

Notation: si la limite d'une fonction \(f\) en \(a\) n'existe pas, on
écrit soit "\(\lim\limits_{x\to a}f(x)\) n'existe pas" ou
"\(\lim\limits_{x\to a}f(x)\not\exists\)".
\end{remarque}


\begin{exemple}
Pour la fonction \(f\) de l'introduction:
\begin{enumerate}
\item \(\lim\limits_{x\to -6}f(x)\) n'existe pas car \(\lim\limits_{x\tog -6}f(x)=\ldots\ldots\neq \lim\limits_{x\tod -6}f(x)=\ldots\ldots\)
\item \(\lim\limits_{x\to -2}f(x)\) n'existe pas car \(\lim\limits_{x\tog -2}f(x)=\ldots\ldots\neq \lim\limits_{x\tod -2}f(x)=\ldots\ldots\)
\item \(\lim\limits_{x\to 6}f(x)\) n'existe pas car \(\lim\limits_{x\tog 6}f(x)=\ldots\ldots\neq \lim\limits_{x\tod 6}f(x)=\ldots\ldots\)
\end{enumerate}
\end{exemple}
\section{Exercices}
\label{sec:orged07fa3}
\begin{exercice}
Analyse le graphique de la fonction suivante.
\begin{center}
\includestandalone[width=0.8\linewidth]{figures/fig12}
\end{center}
\begin{enumerate}
\item Quel est le domaine de \(f\)? \dotfill
\item Détermine graphiquement les limites demandées. Si une limite
n'existe pas, justifie. Si la limite demandée n'apporte pas
d'information pertinente pour la fonction, indique-le.
\begin{center}
\par \setlength{\columnseprule}{0 pt}
          \begin{minipage}[t]{\linewidth}
          \begin{multicols}{3}

\begin{enumerate}
\item \(\lim\limits_{x\to \minf}f(x)=\) \dotfill
\item \(\lim\limits_{x\to -10,5}f(x)=\) \dotfill
\item \(\lim\limits_{x\to -6}f(x)=\) \dotfill
\item \(\lim\limits_{x\to 0}f(x)=\) \dotfill
\item \(\lim\limits_{x\to -3,5}f(x)=\) \dotfill
\item \(\lim\limits_{x\to -2}f(x)=\) \dotfill
\item \(\lim\limits_{x\to -4}f(x)=\) \dotfill
\item \(\lim\limits_{x\to 1}f(x)=\) \dotfill
\item \(\lim\limits_{x\to 1,5}f(x)=\) \dotfill
\item \(\lim\limits_{x\to 3}f(x)=\) \dotfill
\item \(\lim\limits_{x\to +\infty}f(x)=\) \dotfill
\item \(\lim\limits_{x\to 5}f(x)=\) \dotfill
\item \(\lim\limits_{x\to 6}f(x)=\) \dotfill
\item \(\lim\limits_{x\to 4}f(x)=\) \dotfill
\end{enumerate}


\end{multicols}\end{minipage}
\end{center}

\item Donnes les équations des asymptotes.
\end{enumerate}
\end{exercice}

\begin{exercice}
Analyse le graphique de la fonction suivante.
\begin{center}
\includestandalone[width=\linewidth]{figures/fig13}
\end{center}
\begin{enumerate}
\item Quel est le domaine de \(f\)? \dotfill
\item Détermine graphiquement les limites demandées. Si une limite
n'existe pas, justifie. Si la limite demandée n'apporte pas
d'information pertinente pour la fonction, indique-le.
\begin{center}
\par \setlength{\columnseprule}{0 pt}
          \begin{minipage}[t]{\linewidth}
          \begin{multicols}{3}

\begin{enumerate}
\item \(\lim\limits_{x\to \minf}f(x)=\) \dotfill
\item \(\lim\limits_{x\to -5}f(x)=\) \dotfill
\item \(\lim\limits_{x\to -4}f(x)=\) \dotfill
\item \(\lim\limits_{x\to -3}f(x)=\) \dotfill
\item \(\lim\limits_{x\to -3}f(x)=\) \dotfill
\item \(\lim\limits_{x\to 0}f(x)=\) \dotfill
\item \(\lim\limits_{x\to 1}f(x)=\) \dotfill
\item \(\lim\limits_{x\to 2}f(x)=\) \dotfill
\item \(\lim\limits_{x\to 3}f(x)=\) \dotfill
\item \(\lim\limits_{x\to 0,5}f(x)=\) \dotfill
\item \(\lim\limits_{x\to \pinf}f(x)=\) \dotfill
\item \(\lim\limits_{x\to 7}f(x)=\) \dotfill
\end{enumerate}


\end{multicols}\end{minipage}
\end{center}
\item Donnes les équations des asymptotes.
\end{enumerate}
\end{exercice}

\begin{exercice}
Détermine les limites suivantes par lecture graphique. Si une limite
   n'existe pas, justifie. Si la limite demandée n'apporte pas
   d'information pertinente pour la fonction, indique-le.

\par \setlength{\columnseprule}{0 pt}
          \begin{minipage}[t]{\linewidth}
          \begin{multicols}{2}
\begin{center}
\includestandalone[width=0.7\linewidth]{figures/fig14}
\end{center}

a) \(\lim\limits_{x\to \minf}f(x)=\) \dotfill

b) \(\lim\limits_{x\to 1}f(x)=\) \dotfill

c) \(\lim\limits_{x\to \pinf}f(x)=\) \dotfill
\columnbreak

\begin{center}
\includestandalone[width=0.7\linewidth]{figures/fig15}
\end{center}

a) \(\lim\limits_{x\to \minf}f(x)=\) \dotfill

b) \(\lim\limits_{x\to -1}f(x)=\) \dotfill

c) \(\lim\limits_{x\to \pinf}f(x)=\) \dotfill


\end{multicols}\end{minipage}

\par \setlength{\columnseprule}{0 pt}
          \begin{minipage}[t]{\linewidth}
          \begin{multicols}{2}
\begin{center}
\includestandalone[width=0.7\linewidth]{figures/fig16}
\end{center}

a) \(\lim\limits_{x\to \minf}f(x)=\) \dotfill

b) \(\lim\limits_{x\to 0}f(x)=\) \dotfill

c) \(\lim\limits_{x\to \pinf}f(x)=\) \dotfill
\columnbreak

\begin{center}
\includestandalone[width=0.7\linewidth]{figures/fig17}
\end{center}

a) \(\lim\limits_{x\to \minf}f(x)=\) \dotfill

b) \(\lim\limits_{x\to 1}f(x)=\) \dotfill

c) \(\lim\limits_{x\to \pinf}f(x)=\) \dotfill


\end{multicols}\end{minipage}
\end{exercice}

\begin{exercice}
Traduis en termes de limites les situations suivantes. Écris ensuite
les équations des asymptotes.
\par \setlength{\columnseprule}{0 pt}
          \begin{minipage}[t]{\linewidth}
          \begin{multicols}{2}
\includestandalone[width=\linewidth]{figures/fig19}
\includestandalone[width=\linewidth]{figures/fig20}


\end{multicols}\end{minipage}

\par \setlength{\columnseprule}{0 pt}
          \begin{minipage}[t]{\linewidth}
          \begin{multicols}{2}
\includestandalone[width=\linewidth]{figures/fig21}
\includestandalone[width=\linewidth]{figures/fig22}



\end{multicols}\end{minipage}
\end{exercice}

\begin{exercice}
Invente le graphique d'une fonction répondant aux conditions données.

\par \setlength{\columnseprule}{0 pt}
          \begin{minipage}[t]{\linewidth}
          \begin{multicols}{2}
\includestandalone[width=\linewidth]{figures/fig18}
\vspace{1cm}
\begin{center}
\(\dom(f)=\IR\backslash\{1\}\)

\(\lim\limits_{x\to \minf}f(x)=0\)

\(\lim\limits_{x\to 1}f(x)=\pinf\)

\(\lim\limits_{x\to\pinf}f(x)=\minf\)
\end{center}



\end{multicols}\end{minipage}

\par \setlength{\columnseprule}{0 pt}
          \begin{minipage}[t]{\linewidth}
          \begin{multicols}{2}
\includestandalone[width=\linewidth]{figures/fig18}
\vspace{1cm}
\begin{center}
\(\dom(f)=]\minf;5[\)

\(\lim\limits_{x\to \minf}f(x)=\pinf\)

\(\lim\limits_{x\to 2}f(x)=-1\)

\(\lim\limits_{x\to 5}f(x)=-2\)
\end{center}



\end{multicols}\end{minipage}

\par \setlength{\columnseprule}{0 pt}
          \begin{minipage}[t]{\linewidth}
          \begin{multicols}{2}
\includestandalone[width=\linewidth]{figures/fig18}
\vspace{0.1cm}
\begin{center}
\(\dom(f)=]\minf;1[\cup]1;6[\)

\(\lim\limits_{x\to \minf}f(x)=\minf\)

\(\lim\limits_{x\tog 1}f(x)=\minf\)

\(\lim\limits_{x\tod 1}f(x)=2\)

\(\lim\limits_{x\to 6}f(x)=+\infty\)
\end{center}



\end{multicols}\end{minipage}
\end{exercice}
\newpage
\section{Calculs algébriques des limites}
\label{sec:orge214a73}
L'approche graphique des limites n'est pas sans défaut: quand on
analyse le graphique d'une fonction, on n'a pas accès à toute
l'information sur cette fonction. En effet, il n'est pas possible de
représenter sur une feuille de papier le graphique d'une fonction dont
le domaine est, par exemple, \(\IR\). Par exemple, la portion du
graphique de la fonction suivante est trompeur:

\begin{center}
\includestandalone[width=0.5\linewidth]{figures/fig25}
\end{center}

Il s'agit d'une partie du graphique de la fonction
\(f(x)=10\dfrac{\sin(x)}{x}\), où \(\sin(x)\) est la fonction qui à \(x\)
(exprimé en radians) associe son sinus.

Par inspection graphique, on pourrait conjecturer que
\begin{enumerate}
\item \(\lim\limits_{x\to 0}f(x)=\) \dotfill
\item \(\lim\limits_{x\to \minf}f(x)=\) \dotfill
\item \(\lim\limits_{x\to \pinf}f(x)=\) \dotfill
\end{enumerate}

En réalité, le comportement de la fonction est bien différent de ce
l'on pourrait croire. Il suffit pour s'en rendre compte de demander à
des outils numériques de tracer plus largement le graphique de \(f\): si on zoom
en arrière, on observe que nos conjectures sont incorrectes.

Nous avons donc besoin d'une approche plus robuste en ce qui concerne
les limites. Ce sont les calculs algébriques, à partir de l'expression
analytique d'une fonction, qui permettent d'obtenir des réponses
irréfutables, qui permettront de confirmer les conjectures graphiques.
\subsection{Limites des fonctions de référence}
\label{sec:org420a565}

Voici un résumé des limites des fonctions de référence.

\begin{center}
\begin{tabular}{l|c|c|c|}
\cline{2-4}
                                                 & $\lim\limits_{x\to a}f(x),\,\, a\in\dom(f)$ & $\lim\limits_{x\to \minf}f(x)$ & $\lim\limits_{x\to \pinf}f(x)$ \\ \hline
\multicolumn{1}{|l|}{$f(x)=k$}                   &         $k$                                   &      $k$                          &         $k$                       \\ \hline
\multicolumn{1}{|l|}{$f(x)=x$}                   &         $a$                                   &      $\minf$                          &        $\pinf$                        \\ \hline
\multicolumn{1}{|l|}{\rule[-12pt]{0pt}{33pt}$f(x)=\dfrac{1}{x}$}        &         $\dfrac{1}{a}$                                   &  0                              &         $0$                       \\ \hline
\multicolumn{1}{|l|}{$f(x)=|x|$}                 &         $|a|$                                   &           $\pinf$                     &            $\pinf$                    \\ \hline
\multicolumn{1}{|l|}{$f(x)=x^2$}                 &         $a^2$                                   &           $\pinf$                     &             $\pinf$                   \\ \hline
\multicolumn{1}{|l|}{$f(x)=x^n$ avec $n$ pair, $n>0$}   &                                            &                                &                                \\ \hline
\multicolumn{1}{|l|}{$f(x)=\sqrt{x}$}            &         $\sqrt{a}$                                   &            non pertinent                    &        $\pinf$                        \\ \hline
\multicolumn{1}{|l|}{$f(x)=x^3$}                 &         $a^3$                                   &                   $\minf$             &                   $\pinf$             \\ \hline
\multicolumn{1}{|l|}{$f(x)=x^n$ avec $n$ impair} &                                            &                                &                                \\ \hline
\multicolumn{1}{|l|}{$f(x)=\sqrt[3]{x}$}         &         $\sqrt[3]{a}$                                   &            $\minf$                    &                 $\pinf$               \\ \hline
\end{tabular}
\end{center}

De plus: \(\lim\limits_{x\to 0}\dfrac{1}{x}\text{ n'existe pas car
}\lim\limits_{x\tog 0}\dfrac{1}{x}=\minf \text{ et }\lim\limits_{x\tod
0}\dfrac{1}{x}=\pinf\) .


\begin{exercice}
Calcule les limites suivantes:

\par \setlength{\columnseprule}{0 pt}
          \begin{minipage}[t]{\linewidth}
          \begin{multicols}{3}
\(\lim\limits_{x\to -5}x^2=\) \dotfill

\(\lim\limits_{x\to 3 }6=\) \dotfill

\(\lim\limits_{x\to -2}|x|=\) \dotfill

\(\lim\limits_{x\to 4}\sqrt{x}=\) \dotfill

\(\lim\limits_{x\to -2}\sqrt{x}=\) \dotfill

\(\lim\limits_{x\to \pinf}\dfrac{1}{x}=\) \dotfill

\(\lim\limits_{x\to \minf}x^3=\) \dotfill

\(\lim\limits_{x\to \pinf}x=\) \dotfill

\(\lim\limits_{x\to \minf}\sqrt[3]{x}=\) \dotfill



\end{multicols}\end{minipage}
\end{exercice}
\subsection{Calculs avec les symboles \(+\infty\) et \(-\infty\)}
\label{sec:org47d549b}

Faire des calculs avec les symboles \(+\infty\) et \(-\infty\) peut
s'avérer difficile. Que vaut, par exemple \(\dfrac{+\infty}{+\infty}\)? Il
s'avère qu'il n'y a pas de réponse à cette question: la valeur de
cette fraction est indéterminée. Cela veut dire qu'en fonction du
calcul de limite, \(\dfrac{+\infty}{+\infty}\) peut prendre la valeur \(1\),
\(\pi\), \(10\), \(0\) voir même \(+\infty\).

Voici une liste des calculs avec l'infini qu'on rencontrera  dans ce cours
(attention, cette liste n'est pas complète!).
\begin{boite}
Soit \(c\in\IR\) et \(n\in\IZ\). Alors,
\par \setlength{\columnseprule}{0 pt}
          \begin{minipage}[t]{\linewidth}
          \begin{multicols}{2}
\(\pminf + (\pminf) = \dotfill\)


\(\minf - (\pinf) = \dotfill\)

\(\pminf + c=\dotfill\)


\(\pinf\cdot\pinf=\minf\cdot\minf =\dotfill\)

\(\pinf\cdot\minf=\dotfill\)

si \(c\neq 0\), \(\pm\infty\cdot c=\dotfill\)

\((\pinf)^n=\dotfill\)

\((\minf)^n=\dotfill\)


si \(c\neq 0\), \(\dfrac{\pminf}{c}=\dotfill\)


\(\dfrac{c}{\pminf}=\dotfill\)




\end{multicols}\end{minipage}
\end{boite}


Nous rencontrerons dans ce cours les formes indéterminées (\FI{})
suivantes:
\begin{boite}
\[
\frac{c}{0}\hspace{1cm}\frac{0}{0}\hspace{1cm}\frac{\pminf}{\pminf}\hspace{1cm}\pminf\cdot
0\hspace{1cm} \pminf - (\pminf) \hspace{1cm} \pminf + \mp\infty
\]
\end{boite}

\subsection{Règles de calculs}
\label{sec:org9553d11}
Pour calculer des limites de fonctions construites avec les opérations
\(+\), \(-\), \(\cdot\) et \(/\), on dispose des règles de calculs suivantes:
\begin{propriete}
Soit \(f\) et \(g\) deux fonctions et \(a\in\IR\cup\{\minf,\pinf\}\) telles
que les limites \(\lim\limits_{x\to a}f(x)\) et \(\lim\limits_{x\to
a}g(x)\) existent. Alors,
\[
\lim\limits_{x\to a}(f(x)
\stackrel[\displaystyle\cdot]{\displaystyle-}{+} g(x))
=\lim\limits_{x\to a}f(x)
\stackrel[\displaystyle\cdot]{\displaystyle-}{+}
\lim\limits_{x\to a}g(x)\textbf{ sauf si }
\text{le second membre est une \FI{}}
\]
\[
\lim\limits_{x\to a}\dfrac{f(x)}{g(x)}=\dfrac{\lim\limits_{x\to
  a}f(x)}{\lim\limits_{x\to a}g(x)}\textbf{ sauf si }\text{le second
membre est une \FI{}}
\]
\end{propriete}

Nous verrons plus tard des techniques qui permettent de faire des
calculs lorsqu'une \FI{} apparaît.


\begin{exemple}
Calculons les limites suivantes, en un nombre réel.

\begin{enumerate}
\item \(\lim\limits_{x\to 2}x^2-x=\) \dotfill

Explication:\vspace{3cm}
\item \(\lim\limits_{x\to -2}x^3+x-1=\) \dotfill

Explication:\vspace{3cm}
\item \(\lim\limits_{x\to 2}2x^4+3x=\) \dotfill

Explication:\vspace{3cm}
\end{enumerate}
\end{exemple}

\begin{propriete}
Si \(f\) est une fonction polynomiale et \(a\in\IR\), alors
\(\lim\limits_{x\to a}f(x)=\dotfill\)
\end{propriete}

Il nous reste à calculer les limites en \(\pinf\) et \(\minf\).

\begin{exemple}
Essayons d'appliquer la démarche précédente pour calculer la limite
 \(\lim\limits_{x\to \pinf}x^2-x\).
 \vspace{3cm}

On constate qu'il y a une difficulté: on se retrouve avec la
\FI{}  \(\infty-\infty\). \textbf{C'est pour ce genre de situation que
l'énoncé des règles de calculs a des hypothèses.}

Nous pouvons cependant contourner la présence de la \FI{} en
manipulant l'expression \(x^2-x\).
\vspace{5cm}

En conclusion: \(\lim\limits_{x\to \pinf}x^2-x=\) \dotfill
\end{exemple}

\begin{methode}
Pour calculer une limite en \(\pinf\) ou \(\minf\) d'une fonction
polynomiale, on met en évidence le terme du plus haut degré. Ceci
permet de lever l'indétermination \(\infty-\infty\).
\end{methode}

\begin{propriete}
Si \(f\) est une fonction polynomiale alors sa limite en \(\pinf\) (ou
\(\minf\)) est égale à la limite en \(\pinf\) (ou \(\minf\)) de son terme du
plus haut degré. Autrement dit: si \(ax^n\) est le terme du plus haut
degré de \(f\) alors,
\[
\lim\limits_{x\to\pminf}f(x)=\lim\limits_{x\to\pminf}ax^n.
\]
\end{propriete}

Voyons comment appliquer cette propriété avec d'autres exemples.

\begin{exemple}
Calculons les limites suivantes:

\begin{enumerate}
\item \(\lim\limits_{x\to\minf}x^3+x-1=\dotfill\)

\(\lim\limits_{x\to\pinf}x^3+x-1=\dotfill\)

Explication:\vspace{3cm}
\item \(\lim\limits_{x\to\minf}2x^4+3x=\dotfill\)

\(\lim\limits_{x\to\pinf}2x^4+3x=\dotfill\)

Explication:\vspace{3cm}
\end{enumerate}
\end{exemple}

\begin{exercice}
Calcule la limite \(\lim\limits_{x\to\pinf}x^2-2x+1\). Explique ta démarche.
\end{exercice}

\begin{exercice}
Calcule la limite \(\lim\limits_{x\to\minf}(1-x^2)(x+3)\). Explique ta démarche.
\end{exercice}

\begin{exercice}
Calcule rapidement les limites suivantes.
\par \setlength{\columnseprule}{0 pt}
          \begin{minipage}[t]{\linewidth}
          \begin{multicols}{2}
a) \(\lim\limits_{x\to\minf}1-2x-3x^3=\) \dotfill

b) \(\lim\limits_{x\to\pinf}1-2x-3x^3=\) \dotfill

c) \(\lim\limits_{x\to\minf}1+5x^5=\) \dotfill

d) \(\lim\limits_{x\to\pinf}1+5x^5=\) \dotfill

e) \(\lim\limits_{x\to\minf}1-2x-7x^4=\) \dotfill

f) \(\lim\limits_{x\to\pinf}1-2x-7x^4=\) \dotfill

g) \(\lim\limits_{x\to 2}1-2x-7x^4=\) \dotfill

h) \(\lim\limits_{x\to\minf}x(1-x^2)(x+x^3)=\) \dotfill

i) \(\lim\limits_{x\to\pinf}x(1-x^2)(x+x^3)=\) \dotfill

j) \(\lim\limits_{x\to\minf}x(1+2x^2)(1+x-x^4)=\) \dotfill

k) \(\lim\limits_{x\to\pinf}x(1+2x^2)(1+x-x^4)=\) \dotfill

l) \(\lim\limits_{x\to 1}x(1+2x^2)(1+x-x^4)=\) \dotfill


\end{multicols}\end{minipage}
\end{exercice}
\section{Limites des fonctions rationnelles}
\label{sec:org09f5008}
Une fonction rationnelle est un quotient de deux fonctions
polynômiales.

\begin{exercice}
Entoure les expressions qui représentent des fonctions rationnelles.
\begin{center}
\(x\) \hspace{0.5cm}             \(3x^3+6x+\sqrt{x}\) \hspace{0.5cm}  \(\dfrac{x^2+\abs{2x}}{x^2+1}\)
 \hspace{0.5cm}\(\dfrac{1}{x}\) \hspace{0.5cm} \(\dfrac{1}{x^2+1}\) \hspace{0.5cm} \(\dfrac{x^2+3x+1}{x^5+2\sqrt[3]{x}}\)
\end{center}
\end{exercice}

Voici le graphe d'une fonction rationnelle.
\begin{center}
\includestandalone[width=0.8\linewidth]{figures/fig29}
\end{center}

Vous pouvez remarquer que pour cette fonction, le calcul de limites est
pertinent pour tous les réels et pour \(\pinf\) et \(\minf\). Ce sera le
cas pour toutes les fonctions rationnelles. Notre
objectif est de donner des techniques de calcul pour déterminer ces
limites à partir des expressions analytiques. On va distinguer trois
cas pour le calcul de \(\lim\limits_{x\to a}f(x)\):
\begin{enumerate}
\item \(a\) est dans le domaine de \(f\). Dans ce cas, on va constater que
les fonctions rationnelles on la même propriété que les polynômes.
\item \(a\) est une valeur interdite (c'est-à-dire une racine du
dénominateur). Ici on va voir que 5 cas de figure sont à
envisager:

\begin{center}
\includestandalone[width=0.8\linewidth]{figures/fig28}
\end{center}
Le premier cas (graphique de gauche) est le seul où la fonction n'admet pas d'A.V.

\item \(a\) est soit \(\pinf\) soit \(\minf\). Ici on va constater que la mise
en évidence des termes du plus haut degré va nous permettre de
faire des calculs. On pourra ainsi détecter la présence ou non
d'une asymptote horizontale.

\begin{center}
\includestandalone[width=\linewidth]{figures/fig30}
\end{center}
\end{enumerate}


\subsection{Limite en un point du domaine}
\label{sec:org0bdce22}
\begin{propriete}
Si \(f\) est une fonction rationnelle et \(a\in\dom(f)\), alors
\(\lim\limits_{x\to a}f(x)=f(a)\).
\end{propriete}
\begin{remarque}
Cette propriété n'est pas vraie pour toutes les fonctions. Les
fonctions qui satisfont cette propriété sont appellées \emph{fonctions
continues}. Les fonctions de référence,
les fonctions polynômiales et les fonctions rationnelles sont des
exemples de fonctions continues. La fonction de l'exemple introductif
n'est pas une fonction continue car sa limite en \(1\) vaut \(2{,}5\) alors
que \(f(1)=4\).
\end{remarque}

\begin{exemple}
Pour la fonction \(f(x)=\dfrac{x+1}{x^3+x^2+x+1}\),
\(\dom(f)=\IR\backslash\{-1\}\).
\begin{enumerate}
\item \(\lim\limits_{x\to 0}f(x)=f(0)=1\)
\item \(\lim\limits_{x\to 1}f(x)=f(1)=\dfrac{2}{4}=\dfrac{1}{2}\).
\end{enumerate}
\end{exemple}

\subsection{Limite en une valeur interdite}
\label{sec:orgc23dc94}
Nous allons voir comment calculer les limites d'une fonction
rationnelle \(f(x)=\dfrac{P(x)}{Q(x)}\) en une valeur interdite. Pour
rappel, les valeurs interdites sont exactement les racines du
dénominateur: les racine du polynôme \(Q\). Soit \(a\) une racine de
\(Q(x)\). On va disinguer deux cas: \(a\) est aussi une racine du
numérateur et \(a\) n'est pas une racine du numérateur.

\begin{enumerate}
\item \textbf{\(a\) n'est pas une racine du numérateur}
\label{sec:org7b66e57}

Dans cette situation, si on tente de remplacer \(x\) par \(a\) dans
l'expression analytique de \(f\), on se retrouve avec une fraction de la
forme \(\dfrac{c}{0}\), où \(c\neq 0\). Vous avez appris depuis bien
longtemps que cette opération n'est pas permise. Que peut-on dire de
cette situation grâce à la notion de limite?

On aura 3 situations possibles: soit \(\lim\limits_{x\to a}f(x)\) vaut
\(\pinf\) ou \(\minf\), soit la limite n'existe pas.

Voici la démarche générale à adopter dans ce cas de figure:
\begin{methode}
Si \(a\) est une racine du dénominateur mais pas une racine du
numérateur:
\begin{enumerate}
\item on dresse le tableau de signes de \(f\)
\item on calcule les limites à gauche et à droite de \(a\) en utilisant le
tableau de signes
\item si les limites à gauche et à droite sont égales, alors la limite
existe. Sinon, elle n'existe pas.
\end{enumerate}

Dans tous les cas, on se retrouve avec une A.V. d'équation \(x=a\).
\label{org6f5374f}
\end{methode}

\begin{exemple}


Soit la fonction \(f(x)=\dfrac{1}{x+2}\). C'est une fonction rationnelle dont le
domaine est \(\IR\backslash\{-2\}\). En particulier, \(-2\) est une racine du
numérateur.
\par \setlength{\columnseprule}{0 pt}
          \begin{minipage}[t]{\linewidth}
          \begin{multicols}{2}
\begin{center}
\includestandalone[width=0.7\linewidth]{figures/tab4}
\end{center}

On déduit du tableau de signes que:
\[
\lim\limits_{x\tog -2}f(x)=\minf\text{ et }\lim\limits_{x\tod -2}f(x)=\pinf.
\]


\end{multicols}\end{minipage}

En particulier \(\lim\limits_{x\to -2}f(x)\) n'existe pas. De plus, le graphique
de \(f\) a une A.V. d'équation \(x=-2\). On peut vérifier ces résultats avec le
graphique de \(f\).
\label{org81fa43e}
\end{exemple}

\begin{exemple}
Soit la fonction \(f(x)=\dfrac{-1}{(x-1)^2}\). C'est une fonction rationnelle dont le
domaine est \(\IR\backslash\{1\}\). En particulier, \(1\) est une racine du
numérateur.
\par \setlength{\columnseprule}{0 pt}
          \begin{minipage}[t]{\linewidth}
          \begin{multicols}{2}
\begin{center}
\includestandalone[width=0.7\linewidth]{figures/tab5}
\end{center}

On déduit du tableau de signes que:
\[
\lim\limits_{x\tog 1}f(x)=\minf\text{ et }\lim\limits_{x\tod 1}f(x)=\minf.
\]


\end{multicols}\end{minipage}
En particulier \(\lim\limits_{x\to -2}f(x)=\minf\). De plus, le graphique
de \(f\) a une A.V. d'équation \(x=1\). On peut vérifier ces résultats avec le
graphique de \(f\).
\end{exemple}

\begin{exercice}
Soit \(f(x)=\dfrac{1}{(x+1)(x^2-4)}\).

Calcule les limites en \(-2\), \(-1\), \(2\) et \(0\). Interprète graphiquement tes résultats.
\end{exercice}

\begin{exercice}
Soit \(f(x)=\dfrac{x-2}{-x^2+4x}\).

Calcule les limites en \(0\) et \(4\). Interprète graphiquement tes résultats.
\end{exercice}


\item \textbf{\(a\) est une racine du dénominateur}
\label{sec:org23601e8}
Dans cette situation, si on tente de remplacer \(x\) par \(a\) dans
l'expression analytique de \(f\), on se retrouve avec la \FI{} \(\dfrac{0}{0}\).
Que peut-on dire de cette situation grâce à la notion de limite?

La première chose à faire dans cette situation est de simplifier la fraction en
factorisant \textbf{le plus possible}  le numérateur et le dénominateur. Ensuite, on
essaye à nouveau de remplacer \(x\) par \(a\). Deux choses sont
possibles: soit on obtient un nombre \(r\) qui sera la limite, soit on
est face à la \FI{} \(\dfrac{c}{0}\).

\begin{exemple}
Soit \(f(x)=\dfrac{x+1}{x^3+x^2+x+1}\). Son domaine est
\(\IR\backslash\{-1\}\). Si on remplace \(x\) par \(-1\) dans \(f(x)\), on
obtient la \FI{} \(\dfrac{0}{0}\):
\[
\dfrac{-1+1}{(-1)^3+(-1)^2-1+1}=\dfrac{0}{0}.
\]
\(-1\) est donc à la fois une racine du numérateur et du
dénominateur. On peut donc simplifier la fraction par \((x+1)\):
\[
\dfrac{x+1}{x^3+x^2+x+1}=\dfrac{x+1}{(x+1)(x^2+1)}=\dfrac{1}{x^2+1}.
\]

Maintenant, si on remplace \(x\) par \(-1\) dans la fraction simplifiée,
on obtient:
\[
\dfrac{1}{(-1)^2+1}=\dfrac{1}{2}.
\]
Ainsi:
\[
\lim\limits_{x\to -1}f(x)=\dfrac{1}{2}.
\]
Le graphique de \(f\) présente un point creux de coordonnée \((-1,1/2)\).
\end{exemple}

\begin{exemple}
Soit \(f(x)=\dfrac{x+2}{x^2+4x+4}\). Son domaine est
\(\IR\backslash\{-2\}\). Comme \(-2\) est racine du numérateur et du
dénominateur, on peut simplifier la fraction par \((x+2)\):
\[
\dfrac{x+2}{x^2+4x+4}=\dfrac{x+2}{(x+2)^2}=\dfrac{1}{x+2}.
\]
Ici, si on remplace \(x\) par \(-2\) dans la fraction simplifiée, on se
retrouve avec la \FI{} \(\dfrac{c}{0}\). On peut donc appliquer la
méthode du cas précédent: on étudie le signe de la fraction
simplifiée. La suite est exactement comme dans l'exemple
\ref{org81fa43e} .
\end{exemple}

\begin{methode}
Si \(a\) est une racine du numérateur \textbf{et} du dénominateur:

\begin{enumerate}
\item on simplifie au maximum la fraction par \((x-a)\)
\item après simplification, on remplace \(x\) par \(a\):
\begin{enumerate}
\item si le résultat est un nombre \(r\), alors ce nombre est la limite
cherchée. Dans ce cas, on observe graphiquement un point creux de
  coordonnées \((a;r)\)
\item sinon, on a la \FI{} \(\dfrac{c}{0}\). Dans ce cas, on applique la
méthode \ref{org6f5374f} . On observera graphiquement
  une A.V. d'équation \(x=a\)
\end{enumerate}

 \textbf{\uline{Attention}}: si tu obtiens la \FI{} \(\dfrac{0}{0}\) après
simplification, c'est que tu dois encore simplifier par \((x-a)\).
\end{enumerate}
\end{methode}
\end{enumerate}

\subsection{Exercices}
\label{sec:org7459f33}
\begin{exercice}
Soit \(f(x)=\dfrac{x^2+2x-3}{x^2-1}\).
\begin{enumerate}
\item Écris la démarche à appliquer pour calculer la limite de \(f\) aux
valeurs interdites.
\item Applique ta démarche et interprète graphiquement ton résultat.
\end{enumerate}
\label{orgfd57650}
\end{exercice}

\begin{exercice}
Soit \(f(x)=\dfrac{(x-1)^2}{(x-1)(x+2)}\).
\begin{enumerate}
\item Écris la démarche à appliquer pour calculer la limite de \(f\) aux
valeurs interdites.
\item Applique ta démarche et interprète graphiquement ton résultat.
\end{enumerate}
\label{org8c9e388}
\end{exercice}
\begin{exercice}
Soit \(f(x)=\dfrac{2x^2-x-3}{x^3+2x^2+x}\).
\begin{enumerate}
\item Écris la démarche à appliquer pour calculer la limite de \(f\) aux
valeurs interdites.
\item Applique ta démarche et interprète graphiquement ton résultat.
\end{enumerate}
\label{org43c2d77}
\end{exercice}
\begin{exercice}
Invente l'expression analytique d'une fonction rationnelle qui a une
A.V. d'équation \(x=-1\).
\end{exercice}
\begin{exercice}
Invente l'expression analytique d'une fonction rationnelle qui a un point creux
de coordonnée \((2;1)\).
\end{exercice}

\begin{exercice}
Invente l'expression analytique d'une fonction rationnelle qui a une
A.V. d'équation \(x=-1\) \textbf{et} un point creux de coordonnée \((2;1)\).
\end{exercice}

\subsection{Limite en \(\pinf\) ou \(\minf\)}
\label{sec:orgbe59014}
\begin{propriete}
Soit \(f\) une fonction rationnelle. Si \(ax^n\) est le terme du plus haut degré du
numérateur et \(bx^m\) celui du dénominateur, alors
\[
\lim\limits_{x\to\pminf}f(x)=\lim\limits_{x\to\pminf}\dfrac{ax^n}{bx^m}.
\]
\end{propriete}
\begin{exemple}
Soit \(f(x)=\dfrac{5x^2+1}{7x+1}\).

\begin{minipage}{0.35\textwidth}
\includestandalone[width=\linewidth]{figures/fig31}
\end{minipage}
\hspace{0.05\linewidth}
\begin{minipage}[][][t]{0.5\textwidth}
\(\lim\limits_{x\to\pinf}f(x)=\)
\vspace{1cm}

\(\lim\limits_{x\to\minf}f(x)=\)
\vspace{1cm}

Graphiquement, on observe:
\vspace{1cm}
\end{minipage}
\end{exemple}

\begin{exemple}
Soit \(f(x)=\dfrac{x^2+1}{4x^2+1}\).

\begin{minipage}{0.35\textwidth}
\includestandalone[width=\linewidth]{figures/fig32}
\end{minipage}
\hspace{0.05\linewidth}
\begin{minipage}[][][t]{0.5\textwidth}
\(\lim\limits_{x\to\pinf}f(x)=\)
\vspace{1cm}

\(\lim\limits_{x\to\minf}f(x)=\)
\vspace{1cm}

Graphiquement, on observe:
\vspace{1cm}
\end{minipage}
\end{exemple}


\begin{exemple}
Soit \(f(x)=\dfrac{2x^2+1}{x^3+x^2+4}\).

\begin{minipage}{0.35\textwidth}
\includestandalone[width=\linewidth]{figures/fig33}
\end{minipage}
\hspace{0.05\linewidth}
\begin{minipage}[][][t]{0.5\textwidth}
\(\lim\limits_{x\to\pinf}f(x)=\)
\vspace{1cm}

\(\lim\limits_{x\to\minf}f(x)=\)
\vspace{1cm}

Graphiquement, on observe:
\vspace{1cm}
\end{minipage}
\end{exemple}

\begin{tcolorbox}
Pour une fonction rationnelle \(f(x)=\dfrac{ax^n+\cdots}{bx^m+\cdots}\):

\begin{enumerate}
\item si le degré du numérateur > degré du dénominateur: les limites à l'infini
sont infinies. Il n'y a pas d'A.H. et potentiellement, il y aura une A.O.
\item si le degré du numérateur = degré du dénominateur: il y a une A.H. dont
l'équation est \(y=\dfrac{a}{b}\)
\item si le degré du numérateur < degré du dénominateur: il y a une A.H. qui est
\textbf{toujours} d'équation \(y=0\).
\end{enumerate}
\end{tcolorbox}
\begin{exercice}
Calcule les limites à l'infini des fonctions des exercices \ref{orgfd57650},
\ref{org8c9e388} et \ref{org8c9e388}.
\end{exercice}


\chapter{Asymptotes d'une fonction}
\label{sec:org7aa98b0}
\section{Asymptotes verticales}
\label{sec:orga76640a}
\begin{definition}
Soit \(f\) une fonction. La droite d'équation \(x=a\) est une \emph{asymptote verticale}
de \(f\) si
\[\lim\limits_{x\tog a}f(x)=\pminf\text{ ou }\lim\limits_{x\tod a}f(x)=\pminf.\]
\end{definition}

Nous avons déjà recontré à de nombreuses reprises des A.V. et les calculs
associés pour les déterminer. Nous ne ferons donc pas d'exercices calculatoires supplémentaires.

\begin{exercice}
Quelle démarche doit-on appliquer pour déterminer les A.V. d'une fonction rationnelle?
\end{exercice}
\begin{exercice}
Peut-t-on dire qu'une asymptote verticale ne coupe jamais le graphique d'une fonction?
\end{exercice}
\section{Asymptotes non-verticales}
\label{sec:org5864870}
\begin{definition}
Soit \(f\) une fonction. La droite d'équation \(y=mx+p\) est une \emph{asymptote
non-verticale} de \(f\) si
\[
\lim\limits_{x\to\minf}f(x)-(mx+p)=0\text{ ou }\lim\limits_{x\to \pinf}f(x)-(mx+p)=0.
\]
\end{definition}

Les limites de la définition précédente rendent compte du fait que l'écart entre
la fonction et l'asymptote devient de plus en plus proche de \(0\).

Il y a deux types d'asymptotes non-verticales: les asymptotes horizontales (\(m=0\)) et
les asymptotes obliques (\(m\neq 0\)).
\subsection{Asymptotes horizontales}
\label{sec:org5cabcc1}
\begin{propriete}
Soit \(f\) une fonction. La droite d'équation \(y=a\) est une \emph{asymptote horizontale}
de \(f\) si
\[\lim\limits_{x\to\minf}f(x)=a\text{ ou }\lim\limits_{x\to \pinf}f(x)=a.\]
\end{propriete}

Déterminer la présence d'une asymptote horizontale est donc plus simple que de
déterminer la présence d'une asymptote oblique.

Nous avons déjà recontré à de nombreuses reprises des A.V. et les calculs
associés pour les déterminer. Nous ne ferons donc pas d'exercices calculatoires supplémentaires.
\begin{exercice}
Quelle démarche doit-on appliquer pour déterminer les A.H. d'une fonction rationnelle?
\end{exercice}
\begin{exercice}
Peut-t-on dire qu'une asymptote horizontale ne coupe jamais le graphique d'une fonction?
\end{exercice}
\subsection{Asymptotes obliques}
\label{sec:org02a84cd}
Nous allons voir comment en pratique déterminer si une fonction rationnelle a
une asymptote oblique.
Voici quelques fonctions rationnelles.

\begin{center}
\includestandalone[width=\linewidth]{figures/fig35}
\end{center}

Observe les expressions
analytiques de ces fonctions, particulièrement le plus haut degré du
dénominateur et du numérateur. Peux-tu
trouver un critère qui te permet de détecter si une fonction rationnelle a une A.O.?
\vspace{6cm}

Voici un exemple qui illustre comment déterminer l'équation de
l'A.O. d'une fonction rationnelle
\begin{exemple}
Soit \(f(x)=\dfrac{2x^2-1}{x-1}\). Effectuons la division euclidienne de
\(2x^2-1\) par \(x-1\).

\vspace{6cm}

Ce calcul permet de réécrire \(f(x)\) sous la forme \jdot{4cm}. Cette écriture met en
évidence que \(f\) est proche de la droite \(y=\jdot{3cm}\) quand \(x\) est proche de \(\pminf\).

En effet,

\vspace{4cm}


En conclusion, la droite d'équation \(y=\jdot{2cm}\) est l'A.O. de \(f\).
\end{exemple}



\begin{exercice}
Peut-t-on dire qu'une asymptote oblique ne coupe jamais le graphique d'une fonction?
\end{exercice}

\begin{exercice}
Quelle démarche doit-on appliquer pour déterminer l'A.O. d'une fonction rationnelle?
\end{exercice}
\section{Synthèse}
\label{sec:org110a3e9}
La synthèse suivante vous permettra de détecter en un coup d'oeil (ou après de
très brefs calculs) la présence d'asymptotes pour les fonctions rationnelles.
\begin{tcolorbox}
Soit \(f\) une fonction rationnelle.
\begin{enumerate}
\item \(f\) a une A.V. d'équation \(x=a\) si et seulement si\dotfill

\dotfill

\item \(f\) a une A.H. si et seulement si \dotfill

\dotfill

De plus: \dotfill

\dotfill

\item \(f\) a une A.O. si et seulement si \dotfill

\dotfill

\item \(f\) n'a pas d'A.O. ni d'A.H. si et seulement si\dotfill

\dotfill
\end{enumerate}
\end{tcolorbox}
\section{Exercices}
\label{sec:org35f4a2f}
\begin{exercice}
Soit \(f(x)=\dfrac{3x^2-1}{x-3}\). Détermine l'A.O. de \(f\).
\end{exercice}

\begin{exercice}
Soit \(f(x)=\dfrac{2x^2-5x+3}{x-2}\). Détermine l'A.O. de \(f\).
\end{exercice}

\begin{exercice}
Voici le graphique de deux fonctions.

\includestandalone[width=\linewidth]{figures/fig34}

Ces deux fonctions ont une asymptote oblique à droite d'équation
  \(y=\dfrac{x}{2}-1\).

Pour quel(s) graphique(s) peut-on raisonnablement utiliser
l'A.O. pour déterminer l'image de 1000? Explique ton choix et donne
une estimation de(s) image(s) de 1000.
\end{exercice}

\begin{exercice}
Associe les expressions analytiques suivantes aux informations
données, sans faire de calculs.
\par \setlength{\columnseprule}{0 pt}
          \begin{minipage}[t]{\linewidth}
          \begin{multicols}{2}

\uline{Expressions analytiques}

\begin{enumerate}
\item \(f_1(x)=\dfrac{4x^2}{2x^2+1}\)

\item \(f_2(x)=\dfrac{2x}{x-7}\)

\item \(f_3(x)=-2x+5+\dfrac{1}{x-5}\)

\item \(f_4(x)=\dfrac{2x}{x+1}\)

\item \(f_5(x)=x+\dfrac{7}{x^2-4}\)

\item \(f_6(x)=\dfrac{2x^5}{(x+1)(x+10)}\)
\end{enumerate}

\columnbreak

\uline{Informations}

\begin{enumerate}
\item A.V.\(\equiv x=5\) et A.O.\(\equiv y=-2x+5\)
\item deux A.V. en -1 et en -20 et pas d'A.O. ni d'A.H.
\item pas d'A.V. et une A.H.\(\equiv y=2\)
\item deux A.V. en -2 et en 2 et A.O.\(\equiv y=x\)
\item A.V.\(\equiv x=7\) et A.H.\(\equiv y=2\)
\item A.V.\(\equivx=-1\) et A.H.\(\equiv y=2\)
\end{enumerate}



\end{multicols}\end{minipage}
\end{exercice}



\begin{exercice}
Associe chaque expression analytique au graphique correspondant.

\par \setlength{\columnseprule}{0 pt}
          \begin{minipage}[t]{\linewidth}
          \begin{multicols}{4}
\(f_1(x)=\dfrac{3-x^2}{x-1}\)

\(f_2(x)=\dfrac{x}{(x-1)^2}\)

\(f_3(x)=\dfrac{x^2-x-2}{x-1}\)

\(f_4(x)=\dfrac{x^3-x+2}{3x-3}\)


\end{multicols}\end{minipage}
\begin{center}
\includestandalone[width=0.9\linewidth]{figures/fig36}
\end{center}
\end{exercice}
\begin{exercice}
Trouve l'expression analytique d'une fonction dont le graphique a une
A.V. d'équation \(x=1\) et une A.O. d'équation \(y=x-3\).
\end{exercice}
\end{document}
