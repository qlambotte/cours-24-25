% Created 2024-11-10 Sun 17:06
% Intended LaTeX compiler: pdflatex
\documentclass[a4paper,12pt,singlepage]{report}
\usepackage[utf8]{inputenc}
\usepackage[T1]{fontenc}
\usepackage{graphicx}
\usepackage{longtable}
\usepackage{wrapfig}
\usepackage{rotating}
\usepackage[normalem]{ulem}
\usepackage{amsmath}
\usepackage{amssymb}
\usepackage{capt-of}
\usepackage{hyperref}
\usepackage{my-conf}
\renewcommand{\arraystretch}{1.2}
\newcommand{\IR}{\mathbb{R}}
\newcommand{\IZ}{\mathbb{Z}}
\newcommand{\IQ}{\mathbb{Q}}

\newcommand{\IN}{\mathbb{N}}
\renewcommand{\emptyset}{\varnothing}
\newcommand{\dom}{\mathrm{dom}}
\newcommand{\im}{\mathrm{im}}
\newcommand{\jdot}[1]{ \makebox[#1]{\dotfill}}
\newcommand{\tog}{\stackrel[<]{}{\to}}
\newcommand{\tod}{\stackrel[>]{}{\to}}
\newcommand{\pinf}{+\infty}
\newcommand{\minf}{-\infty}
\newcommand{\pminf}{\pm\infty}
\newcommand{\mpinf}{\mp\infty}
\newcommand{\FI}{\textbf{F.I.}}
\newcommand{\abs}[1]{|#1|}
\newcommand{\rad}{\text{rad}}
\author{Quentin Lambotte}
\date{2024-11-10}
\title{Fonctions exponentielles et logarithmes\\\medskip
\large 6UAA4}
\hypersetup{
 pdfauthor={Quentin Lambotte},
 pdftitle={Fonctions exponentielles et logarithmes},
 pdfkeywords={},
 pdfsubject={notes de cours de l'uaa2, 5GTT, math 4p},
 pdfcreator={Emacs 29.3 (Org mode 9.6.15)}, 
 pdflang={French}}
\begin{document}

\begin{titlepage}
\begin{center}
\vspace{2cm}
\Huge
\textbf{6UAA4: Fonctions exponentielles et logarithmes}
\vspace{2cm}

\begin{center}
\includestandalone[width=1\linewidth]{figures/fig6}
\end{center}
\end{center}
\end{titlepage}



\singlespacing
\small

\uline{\textbf{Remarque:}} voici une autre version de l'image de la première page. Vous aurez
l'occasion d'utiliser ces deux figures comme une synthèse. \textbf{Attention!}
certaines informations sur les figures sont redondantes; à vous de trouver les
points communs aux figures au fur et à mesure des deux premiers chapitres de
cette UAA.
\begin{center}
\includestandalone[width=0.8\linewidth]{figures/fig7}
\end{center}

\uline{Compétences à développer dans ce chapitre :}
\begin{itemize}
\item Connaître
\begin{itemize}
\item Démontrer des propriétés des fonctions logarithmes.
\item Comparer les croissances des fonctions exponentielles, logarithmes et
puissances sur \(\IR^{>0}\)
\end{itemize}
\item Appliquer
\begin{itemize}
\item Résoudre une équation exponentielle simple.
\item Résoudre une équation logarithmique simple.
\item Calculer des limites, des dérivées et des primitives de fonctions
exponentielles et logarithmes.
\item Extraire des informations d’un graphique en coordonnées logarithmique ou
semi-logarithmique.
\end{itemize}
\item Transférer
\begin{itemize}
\item Choisir une échelle adéquate pour représenter les données d’un problème.
\item Utiliser une fonction logarithme ou exponentielle pour résoudre un problème.
\item Modéliser un nuage de points par une fonction exponentielle.
\item Reconnaitre, parmi tous ceux déjà rencontrés, le modèle adéquat à la
situation proposée.
\end{itemize}
\end{itemize}
\onehalfspacing
\chapter{Fonctions exponentielles}
\label{sec:orgeb3fa81}
\section{Croissance exponentielle}
\label{sec:org8bf92f9}
\subsection{Propagation d'un secret}
\label{sec:orge85e96c}
Imaginons deux élèves des Ursulines, Laura et Diego. Ils partagent un secret au
jour 1. Le lendemain, Diego raconte ce secret à deux nouvelles personnes. Le
surlendemain, ces deux personnes racontent le secret à deux autres \uline{nouvelles}
personnes, chacune. Et ainsi de suite.

De cette situation, on peut se poser quelques questions:
\begin{itemize}
\item dans combien de temps les élèves de l'école seront au courant du secret?
\item dans combien de temps l'ensemble des montois sera au courant du secret?
\item on peut se poser la même question pour le Hainaut, la Belgique, et le monde.
\end{itemize}

Le fait surprenant est qu'il faut un peu plus d'un mois pour que la planète
entière soit au courant du secret\ldots{} Essayons de modéliser la situation pour
comprendre cette curieuse propriété.

La première étape, toujours utile, est d'essayer de représenter la situation
avec un dessin:

\vspace{10cm}

Maintenant, on va introduire des notations pour formaliser le dessin. Pour
chaque jour \(n\), on va noter \(u_n\) le nombre de personnes qui connaît le
secret.

Faisons quelques calculs:
\par \setlength{\columnseprule}{0 pt}
          \begin{minipage}[t]{\linewidth}
          \begin{multicols}{3}
\begin{itemize}
\item \(u_0=\dotfill\)
\item \(u_{1}=\dotfill\)
\item \(u_{2}=\dotfill\)
\item \(u_{3}=\dotfill\)
\item \(u_{4}=\dotfill\)
\item \(u_{5}=\dotfill\)
\item \(u_{6}=\dotfill\)
\item \(u_{7}=\dotfill\)
\item \(u_{8}=\dotfill\)
\end{itemize}


\end{multicols}\end{minipage}

Vous pouvez observer une relation entre les différentes valeur de \(u_{n}\). Cette
relation est
\[
u_{n}=\jdot{4cm}
\]

Cette relation peut être transformée comme ceci:
\[
u_{n}=\jdot{4cm}
\]

Revenons à nos questions:

\begin{itemize}
\item Il y a environ 1800 élèves aux Ursulines. Après combien de jours toute l'école
sera au courant du secret?\dotfill
\item Il y a environ 96 000 habitants à Mons. Après combien de jours tout Mons
sera au courant du secret?\dotfill
\item Il y a environ 1 340 000 habitants dans le Hainaut. Après combien de jours
tout le Hainaut sera au courant du secret?\dotfill
\item Il y a environ 11 600 000 Belges. Après combien de jours toute la Belgique
sera au courant du secret?\dotfill
\item Il y a environ 8 000 000 000 Terriens. Après combien de jours toute la Terre
sera au courant du secret?\dotfill
\end{itemize}

Remplacez dans la situation \emph{secret} par \emph{virus COVID-19} et vous comprendrez
qu'un confinement permet d'éviter des scénarios catastrophe\ldots{}

Vous avez ici un premier exemple d'un phénomène de \emph{croissance exponentielle}:
la progression du nombre \(u_{n}\) de personnes connaissant le secret croit de plus
en plus vite.
\subsection{Placements à intérêts simples et composés}
\label{sec:org94464f8}

Vous avez peut-être un compte d'épargne. Savez-vous comment ce compte
fonctionne? Nous allons voir dans les deux exemples suivants le méchanisme qui
permet de modéliser comment votre argent grandit sur un compte d'épargne.

Il existe deux manières de faire croître de l'argent: par intérêts simples et
par intérêts composés.

Dans les intérêts simples, la somme gagnée après chaque période dépend
uniquement du montant déposé initialement. Dans les intérêts composés, le somme
gagnée après une période dépend de la somme obtenue à la période précédente;
autrement dit, dans les intérêts composés, l'intérêt est réévalué à chaque
période.

Supposons que vous déposez 1000\texteuro{} sur un compte d'épargne avec un taux \textbf{annuel}
de \(10\%\). Nous allons déterminer l'évolution de ce montant sur dix ans, lorsque
l'intérêt est simple et lorsque l'intérêt est composé.

\begin{enumerate}
\item Placement à intérêt simple.
\label{sec:org5ed7ffd}

Pour rappel l'intérêt simple consiste à calculer une fois l'intérêt puis de
l'ajouter de période en période.

Ici l'intérêt est de \(I=100\). On ajoute donc \(100\) \texteuro{} chaque année.

\begin{center}
\begin{tabular}{|l|p{1cm}|p{1cm}|p{1cm}|p{1cm}|p{1cm}|p{1cm}|p{1cm}|p{1cm}|p{1cm}|p{1cm}|p{1cm}|}
\hline
années & 0 & 1 & 2 & 3 & 4 & 5 & 6 & 7 & 8 & 9 & 10\\[0pt]
\hline
montant & 1000 & 1100 & 1200 & 1300 & 1400 & 1500 & 1600 & 1700 & 1800 & 1900 & 2000\\[0pt]
\hline
\end{tabular}
\end{center}

Voici une représentation graphique de la situation.

\begin{center}
\includestandalone[width=0.8\linewidth]{figures/fig1}
\end{center}

\item Placement à intérêt composés.
\label{sec:org081701f}

Ici l'intérêt est recalculé à chaque période.

\begin{center}
\begin{tabular}{|l|p{0.8cm}|p{0.8cm}|p{0.8cm}|p{0.8cm}|p{1cm}|p{1.1cm}|p{1.1cm}|p{1.1cm}|p{1.1cm}|p{1.1cm}|p{1.1cm}|}
\hline
années & 0 & 1 & 2 & 3 & 4 & 5 & 6 & 7 & 8 & 9 & 10\\[0pt]
\hline
montant & 1000 & 1100 & 1210 & 1331 & 1464,1 & 1610,51 & 1771,56 & 1948,72 & 2143,59 & 2357,95 & 2593,74\\[0pt]
\hline
\end{tabular}
\end{center}


Voici une représentation graphique de la situation.

\begin{center}
\includestandalone[width=0.8\linewidth]{figures/fig8}
\end{center}

\item Comparaison des deux placements
\label{sec:orgab8a7b4}

Il est utile de comparer les deux placements. Voici un graphique reprenant les
deux situations.

\begin{center}
\includestandalone[width=0.8\linewidth]{figures/fig9}
\end{center}

Que constates-tu? \dotfill

\dotfill

\dotfill


Faisons des observations plus profondes en modélisant comme pour le premier
exemple.

On note \(C_n\) le capital à la période \(n\). Alors:
\begin{itemize}
\item À intérêts simples: \(C_n=\dotfill\)
\item À intérêts composés: \(C_n=\dotfill\)
\end{itemize}

\item Exercice
\label{sec:org21c9c99}

Traduis chacune des augmentations (ou diminutions) en pourcentage suivantes à
l'aide d'un coefficient multiplicateur:

\begin{itemize}
\item Une augmentation de \(20\%\) correspond à une multiplication par\dotfill
\item Une augmentation de \(5\%\) correspond à une multiplication par\dotfill
\item Une diminution de \(15\%\) correspond à une multiplication par\dotfill
\item Une diminution de \(3\%\) correspond à une multiplication par\dotfill
\end{itemize}
\end{enumerate}


\subsection{Définition de la croissance exponentielle.}
\label{sec:orgf444d87}

\begin{definition}
Etant donnée une fonction \(f\), on dira qu'elle a une croissance exponentielle
si le taux de croissance est constant pour des intervalles de longueur
constante. Autrement dit:

lorsque des valeur de \(x\) sont les termes d'une \textbf{suite arithmétique}, les valeurs de
\(f(x)\) sont les termes d'une \textbf{suite géométrique}.
\end{definition}

Nous allons comprendre cette définition à l'aide de la construction de la
fonction \(f(x)=2^x\).
\section{Construction et définition de la fonction exponentielle}
\label{sec:org6d50c06}
\subsection{Construction de \(f(x)=2^{x}\)}
\label{sec:org173859f}

Pour construire la fonction \(f(x)=2^x\), on sa se placer dans une situation
concrète. La situation est la suivante: vous placez \(1\) euro sur un compte
d'épargne où le taux \textbf{annuel} est de 100\(\%\). On a vu dans un des exemples
introductifs que l'évolution de capital par année est une suite géométrique, de
raison \(1+100\%=1+1=2\). Donc si on s'intéresse à l'évolution de votre euros sur
les 20 prochaines années, on sait qu'il sera calculé grâce à la formule
\[
C_n=1\cdot 2^n=2^n.
\]

Voici une représentation de cette évolution.

\begin{center}
\includestandalone[width=0.6\linewidth]{figures/fig10}
\end{center}

Ce graphique montre une forte croissance: après 20 ans, on devient millionnaire!

On va se concentrer sur le début du graphique, disons les 4 premières années.



\par \setlength{\columnseprule}{0 pt}
          \begin{minipage}[t]{\linewidth}
          \begin{multicols}{2}
\includestandalone{figures/fig11}

\columnbreak

Notre objectif va être de combler les trous sur le graphique, de sorte à ce
que chaque valeur de \(x\) puisse avoir une image.

Pour y arriver on va faire un raisonnement financier: plutôt que d'envisager
l'évolution de l'argent par années, on va l'envisager par demi-années.

On se base sur la formule \(C_n=2^n\). Ici \(n\) représente le nombre
d'années. Replaçons \(n\) par \(n/2\) dans la formule. On crée ainsi une nouvelle
suite qui va modéliser l'évolution de l'argent par demi année.

Le graphique de cette suite est à la page suivante.




\end{multicols}\end{minipage}
\begin{center}
\includestandalone{figures/fig12}
\end{center}

On peut même poursuivre le raisonnement en divisant les années en plus petites
parties plus petites (en tiers, quart, etc). Voici ce que cela donne avec des
quarts, des 8e etc.:

\par \setlength{\columnseprule}{0 pt}
          \begin{minipage}[t]{\linewidth}
          \begin{multicols}{2}
\includestandalone[width=0.8\linewidth]{figures/fig13}

\includestandalone[width=0.8\linewidth]{figures/fig15}
\columnbreak

\includestandalone[width=0.8\linewidth]{figures/fig14}

\includestandalone[width=0.8\linewidth]{figures/fig16}


\end{multicols}\end{minipage}

Vous constatez que plus on subdivise une année, plus les trous cont comblés; si
bien que sur le dernier graphique, on ne discerne plus de trous entre chaque
point et on observe une courbe.

C'est ainsi que la fonction \(f(x)=2^x\) peut être construite à partir
d'opérations que vous connaissez: les exposants fractionnaires.

La fonction \(f(x)=2^x\) est bien une fonction à croissance
exponentielle. Reprenons les graphiques de la construction:
\begin{center}
\includestandalone[width=\linewidth]{figures/fig12-2}
\end{center}

\newpage
\subsection{Définition}
\label{sec:org64a06f6}
\begin{definition}
Soit \(a\in\IR^{>0}\backslash\{1\}\). La fonction \emph{exponentielle de base \(a\)} est
l'unique fonction \(f\) qui satisfait les deux conditions suivantes:
\begin{itemize}
\item \(f\) prolonge les exposant rationnels de base \(a\);
\item \(f\) est dérivable en \(0\).
\end{itemize}

L'expression analytique de \(f\) sera notée \(f(x)=a^x\).
\end{definition}

Pourquoi demande-t-on que la base soit dans \(\IR^{>0}\backslash\{1\}\)?

\dotfill

\dotfill

\dotfill

\dotfill

\dotfill

\dotfill

\begin{exercice}
Entoure les fonctions qui sont exponentielles.

\par \setlength{\columnseprule}{0 pt}
          \begin{minipage}[t]{\linewidth}
          \begin{multicols}{5}

\(f(x)=2^x\)

\columnbreak

\(f(x)=\left(\dfrac{1}{2}\right)^x\)

\columnbreak

\(f(x)=(-3)^x\)

\columnbreak

\(f(x)=3,25^x\)

\columnbreak

\(f(x)=x^3\)




\end{multicols}\end{minipage}
\end{exercice}

La définition de la fonction exponentielle de base \(a\) dit qu'elle prolonge les
exposants rationnels de base \(a\). En conséquence, on a la propriété importante
suivante:
\begin{propriete}
Soit \(a\in\IR^{>0}\backslash\{1\}\). Alors quels que soient \(m,p\in\IR\):
\begin{itemize}
\item \(a^0=1\)
\item \(a^{m+p}=a^ma^p\)
\item \((a^m)^p=a^{mp}\)
\item \(\dfrac{a^m}{a^p}=a^{m-p}\) (autrement dit: l'exponentielle a une croissance
exponentielle)
\end{itemize}
\end{propriete}
\newpage
\section{Propriétés et caractéristiques graphiques}
\label{sec:orga6d32f9}
\begin{exercice}
Trace les graphiques des fonctions exponentielles suivantes.

\begin{center}
\begin{tabular}{|l|p{2cm}|p{2cm}|p{3cm}|p{3cm}|}
\hline
\(x\) & \(f_1(x)=2^x\) & \(f_2(x)=3^x\) & \(f_3(x)=\left(\dfrac{1}{2}\right)^x\) & \(f_4(x)=\left(\dfrac{1}{3}\right)^x\)\\[0pt]
\hline
\(-3\) &  &  &  & \\[0pt]
\hline
\(-2\) &  &  &  & \\[0pt]
\hline
\(-1\) &  &  &  & \\[0pt]
\hline
\(0\) &  &  &  & \\[0pt]
\hline
\(1\) &  &  &  & \\[0pt]
\hline
\(2\) &  &  &  & \\[0pt]
\hline
\(3\) &  &  &  & \\[0pt]
\hline
\end{tabular}
\end{center}


\begin{center}
\includestandalone[width=0.8\linewidth]{figures/fig17}
\end{center}

\begin{observation}
Sur base des graphiques, réponds aux questions suivantes.

Que vaut \(a^0\)?\dotfill

Que vaut \(a^1\)?\dotfill

Nous remarquons que des points particuliers des fonctions exponentielles sont:
\dotfill
\end{observation}
\end{exercice}
\newpage
Voici les allures graphiques que peuvent prendre les exponentielles.
\begin{center}
\includestandalone[width=\linewidth]{figures/fig18}
\end{center}

On peut extraire de ces graphiques les propriétés suivantes.
\begin{propriete}
Soit \(a\in\IR^{>0}\backslash\{1\}\). On note \(f\) l'exponentielle de
base \(a\). Alors,

\par \setlength{\columnseprule}{2 pt}
          \begin{minipage}[t]{\linewidth}
          \begin{multicols}{2}
\begin{itemize}
\item \(\dom{f}=\dotfill\)
\item \(\im{f}=\dotfill\)
\item Racines=\dotfill
\item o.à.o=\dotfill
\item \(G_f\cap x=\dotfill\)
\item \(G_f\cap y=\dotfill\)
\item AV\dotfill
\item AO\dotfill
\item Points particuliers:\dotfill
\item Concavité \dotfill
\end{itemize}


\end{multicols}\end{minipage}

De plus,
\par \setlength{\columnseprule}{2 pt}
          \begin{minipage}[t]{\linewidth}
          \begin{multicols}{2}
\uline{\textbf{si \(a>1\)}}
\begin{itemize}
\item \(f\) est \dotfill
\item \(\lim\limits_{x\to\infty}f(x)=\dotfill\)
\item \(\lim\limits_{x\to-x\infty}f(x)=\dotfill\)
\item \(AH_{-\infty}\equiv\dotfill\)
\end{itemize}

\columnbreak
 \uline{\textbf{si \(0<a<1\)}}
\begin{itemize}
\item \(f\) est \dotfill
\item \(\lim\limits_{x\to\infty}f(x)=\dotfill\)
\item \(\lim\limits_{x\to-\infty}f(x)=\dotfill\)
\item \(AH_{+\infty}\equiv\dotfill\)
\end{itemize}


\end{multicols}\end{minipage}
\end{propriete}

Vous avez de plus remarqué une symétrie des graphiques ci-dessus.
\begin{propriete}
Soit \(a\in\IR^{>0}\backslash\{1\}\). Alors le graphique de
\(f(x)=\dfrac{1}{a^x}\) est obtenu à partir du graphique de \(g(x)=a^x\)
par une symétrie d'axe \(y\).
\end{propriete}
\section{Multiple d'une fonction exponentielle}
\label{sec:orgcaebea3}
Généralement, dans les problèmes de modélisation, ce sont des multiples des
fonctions exponentielles qui sont le plus fréquemment rencontrés.
\begin{definition}
Un multiple d'une fonction exponentielle est une fonction dont
l'expression analytique est de la forme \(ka^x\), où
\(a\in\IR^{>0}\backslash\{1\}\) et \(k\in\IR_0\).
\end{definition}
Voici quelques exemples.
\vspace{1cm}

\includestandalone[width=\linewidth]{figures/fig19}
\vspace{1cm}

\par \setlength{\columnseprule}{0 pt}
          \begin{minipage}[t]{\linewidth}
          \begin{multicols}{2}
\(f(0)=\dotfill\)

\(f(1)=\dotfill\)

\(f\) est \dotfill

\(f(0)=\dotfill\)

\(f(1)=\dotfill\)

\(f\) est \dotfill


\end{multicols}\end{minipage}

\vspace{1cm}
\begin{center}
\includestandalone[width=0.8\linewidth]{figures/fig20}
\end{center}
\vspace{1cm}

\par \setlength{\columnseprule}{0 pt}
          \begin{minipage}[t]{\linewidth}
          \begin{multicols}{2}
\(f(0)=\dotfill\)

\(f(1)=\dotfill\)

\(f\) est \dotfill

\(f(0)=\dotfill\)

\(f(1)=\dotfill\)

\(f\) est \dotfill


\end{multicols}\end{minipage}

\section{Exercices}
\label{sec:org563f14c}
\begin{exercice}
Compare graphiquement les fonctions \(2^x\) et \(x^2\).
\end{exercice}

\begin{exercice}
Détermine l’expression analytique des fonctions exponentielles ci-dessous.

\includestandalone{figures/fig21}
\end{exercice}

\begin{exercice}
Parmi les fonctions suivantes, quelles sont celles qui sont des fonctions
exponentielles ou multiples de fonctions exponentielles ?

\par \setlength{\columnseprule}{0 pt}
          \begin{minipage}[t]{\linewidth}
          \begin{multicols}{5}

\(f(x)=1^x\)

\columnbreak

\(f(x)=-\left(\dfrac{1}{3}\right)^x\)

\columnbreak

\(f(x)=(5^{-1})^x\)

\columnbreak

\(f(x)=2\cdot x^{\frac{5}{3}}\)

\columnbreak

\(f(x)=4\left(-\dfrac{1}{3}\right)^x\)




\end{multicols}\end{minipage}
\end{exercice}

\begin{exercice}
Pour chaque tableau, précise s’il représente une situation de croissance
exponentielle. Explique tes choix.

\par \setlength{\columnseprule}{0 pt}
          \begin{minipage}[t]{\linewidth}
          \begin{multicols}{3}

\begin{center}
\begin{tabular}{|c|c|}
\hline
\(x\) & \(y\)\\[0pt]
\hline
0 & 10\\[0pt]
\hline
2 & 100\\[0pt]
\hline
4 & 1000\\[0pt]
\hline
5 & 10000\\[0pt]
\hline
\end{tabular}
\end{center}

\begin{center}
\begin{tabular}{|c|c|}
\hline
\(x\) & \(y\)\\[0pt]
\hline
-3 & 25\\[0pt]
\hline
0 & 5\\[0pt]
\hline
3 & 1\\[0pt]
\hline
6 & 0,2\\[0pt]
\hline
\end{tabular}
\end{center}

\begin{center}
\begin{tabular}{|c|c|}
\hline
\(x\) & \(y\)\\[0pt]
\hline
-3 & 3\\[0pt]
\hline
-2 & 9\\[0pt]
\hline
-1 & 81\\[0pt]
\hline
9 & 561\\[0pt]
\hline
\end{tabular}
\end{center}


\end{multicols}\end{minipage}
\end{exercice}

\begin{exercice}
En 1850, Bourgville comptait 100 habitants. Depuis lors, la population de cette
localité n'a cessé d'augmenter ; elle double tous les dix ans.
\begin{itemize}
\item Modélise l'évolution de cette population en donnant l’expression analytique
de la fonction qui représente le nombre d’habitants à Bourgville \(x\) années
après l’année 1850.
\item Trouve combien Bourgville comptait d'habitants en 1900, en 1950, en 2000
et en 2003
\end{itemize}
\end{exercice}

\begin{exercice}
A l’occasion du salon de l’automobile en janvier 2017, Jules achète une nouvelle
voiture full options au prix de 29850 euros.
\begin{itemize}
\item Sachant qu’une voiture se déprécie annuellement d’environ 15\(\%\)
pendant les 6 premières années, modélise l’évolution du prix du
véhicule par une fonction qui permet de calculer la valeur de la
voiture \(x\) années après l’achat.
\item Si la dépréciation continue au même taux, quelle sera la valeur de
la voiture 10 ans après l’achat.
\item Trace le graphique de cette fonction exponentielle. Peut-on parler
de « bon placement » lors de l’achat d’une voiture ?
\end{itemize}
\end{exercice}

\begin{exercice}
Dans une famille de lapins particulièrement prolifique, le nombre
d'individus double tous les sept jours. Supposons que cette famille comprenne
actuellement 100 lapins.
\begin{itemize}
\item Donne l’expression analytique de la fonction qui modélise
l’évolution de cette population de lapins en fonction du temps
exprimé en jours.
\item Combien y aura-t-il de lapins dans un mois (30 jours), dans deux
mois et dans trois mois ?
\end{itemize}
\end{exercice}

\begin{exercice}
Dans un certain pays en pleine crise économique, la monnaie se déprécie
annuellement de 8,5 \(\%\). L'unité monétaire de ce pays étant le sou,
quelle sera la "valeur réelle" d'un billet de 1000 sous
\begin{itemize}
\item dans 5 ans et dans 10 ans ?
\item Donne l’expression analytique de la fonction qui représente le
décroissement de la valeur d’un billet de 1000 sous en fonction du temps.
\end{itemize}
\end{exercice}
\newpage
\section{Similitude des graphiques des fonctions exponentielles}
\label{sec:orgcdb179c}
Les graphes des courbes exponentielles sont semblables.

Voici dix fonctions exponentielles:

\begin{center}
\includestandalone[width=0.8\linewidth]{figures/fig22}
\end{center}

Cette similitudes des graphique ammène à penser que les graphiques des fonctions
exponentielles sont des transformés les uns des autres.

Par exemple, il est possible de produire le graphique de n'importe quelle
fonction exponentielle en tranformant le graphique de \(2^x\).

\begin{propriete}
Soit \(a\in \IR^{>0}\backslash\{1\}\). Alors il existe \(k\in\IR\) tel que pour tout
\(x\in\IR\),
\[
a^x=2^{kx}.
\]
De plus on peut remplacer dans l'affirmation précédente \(2\) par n'importe quel
autre nombre dans \(\IR^{>0}\backslash\{1\}\).
\end{propriete}

Ainsi, il serait intéressant de \textbf{choisir} un représentant des fonctions
exponentielles pour les étudier.

On pourrait par exemple choisir \(2^x\) pour le reste de l'étude des fonctions
exponentielles.

Mais bien qu'il s'agisse d'un choix raisonnable, nous allons choisir un autre
représentant des fonctions exponentielle dont la base a une propriété
remarquable. Nous nommerons \(e\) ce nombre, une constante qui porte l'initiale du
nom du mathématicien Leonard Euler. Ce nombre est omniprésent en mathématique.

\section{Le nombre \(e\) et l'exponentielle népérienne}
\label{sec:org50b8b1b}
\subsection{Dérivabilité des exponentielles}
\label{sec:orgf5c40c3}

Dans la définition des fonctions exponentielle, on fait \textbf{l'hypothèse} que la
fonction exponentielle est dérivable en \(0\). C'est-à-dire, que pour une
fonctions exponentielle \(f\), de base \(a\), le nombre \(f'(0)\) existe. Pour rappel, ce nombre
est définicomme la limite suivante:

\[
f'(0)=\lim_{h\to 0} \frac{f(0+h)-f(0)}{h}=\lim_{h\to 0} \frac{a^{h}-1}{h}.
\]

Ce nombre représente la pente de la tangente en graphe de \(f\) au point \((0,1)\).

Nous allons voir que l'existence de \(f'(0)\) aura une conséquence remarquable:
non seulement les exponentielles sont dérivables en 0, mais elles le sont aussi
partout sur leur domaine.

Pour rappel, être dérivable en un point \(x\) signifie que la limite suivante
existe:
\[
f'(x)=\lim_{h\to 0} \frac{f(x+h)-f(x)}{h}.
\]

\begin{propriete}
Soit \(a\in \IR^{>0}\backslash\{1\}\). Alors la fonction \(f(x)=a^x\) est dérivable
sur \(\IR\). De plus,
\[
f'(x)=f'(0)\cdot a^x.
\]
Autrement dit, la dérivée de l'exponentielle n'est rien d'autre qu'un multiple
de l'exponentielle elle-même!
\end{propriete}

\begin{proof}
On se fixe \(a\in \IR^{>0}\backslash\{1\}\) et on note \(f\) l'exponentielle de base
\(a\). Alors, quels que soient \(x,h\in\IR\):

\begin{alignat*}{3}
\frac{f(x+h)-f(x)}{h}&= \frac{a^{x+h}-a^x}{h}&\hspace{0.5cm}&\text{(définition de $f$)}\\
&=\frac{a^xa^h-a^x}{h}&&\text{(propriété de l'exponentielle)}\\
&=a^x\frac{a^h-1}{h}&&\text{(mise en évidence)}
\end{alignat*}

Donc

\begin{alignat*}{3}
f'(x)&=\lim_{h\to 0} \frac{f(x+h)-f(x)}{h}&\hspace{0.5cm}&\\
&=\lim_{h\to 0} a^x\frac{a^h-1}{h}&&\text{(par les calculs précédents)}\\
&=a^x\lim_{h\to 0} \frac{a^h-1}{h} &&\text{($a^x$ est indépendant de $h$)}\\
&=a^x f'(0)&&\text{(par définition de l'exponentielle,$\lim_{h\to 0} \frac{a^h-1}{h}$ existe.)}
\end{alignat*}

Donc l'exponentielle est bien dérivable partout sur son domaine.
\end{proof}

Nous donnerons dans un autre chapitre un moyen de calculer \(f'(0)\).

Cette propriété de dérivabilité de l'exponentielle est importante. Nous allons
choisir un représentant des exponentielles à l'aide de cette propriété en
cherchant une base \(a\) telle que pour \(f(x)=a^x\), on ait \(f'(0)=1\). On aura
ainsi trouvé une fonction égale à sa dérivée!

\subsection{Définition du nombre \(e\)}
\label{sec:org83963b9}

Nous cherchons une base \(a\) telle que pour \(f(x)=a^x\), on ait \(f'(0)=1\). On sait
que
\[
f'(0)=\lim_{h\to 0} \frac{a^{h}-1}{h}=1.
\]

Ceci veut dire que la fraction \(\dfrac{a^{h}-1}{h}\) est proche de \(1\) lorsque
\(h\) est proche de \(0\):
\begin{alignat*}{3}
&\dfrac{a^{h}-1}{h}\approx 1\\
\Leftrightarrow& a^h-1\approx h\\
\Leftrightarrow&a^h\approx 1+h\\
\Leftrightarrow&a\approx (1+h)^{1/h}
\end{alignat*}
Nous avons donc un moyen d'estimer le nombre recherché en prenant des valeur de
\(h\) petites:
\begin{center}
\begin{tabular}{|p{3cm}|p{3cm}|}
\hline
\(h\) & \((1+h)^{1/h}\)\\[0pt]
\hline
0,1 & \\[0pt]
\hline
0,01 & \\[0pt]
\hline
0,001 & \\[0pt]
\hline
0,0001 & \\[0pt]
\hline
0,0000001 & \\[0pt]
\hline
\end{tabular}
\end{center}

\begin{definition}
Le nombre d'Euler, noté \(e\), est défini par l'égalité suivante:
\[
e=\lim_{h\to 0}(1+h)^{1/h}=2,718281828459\ldots.
\]
\end{definition}

C'est une constante très importante. Ce n'est pas un nombre rationnel; on dit
qu'il est irrationnel. Il a même une propriété forte: c'est un nombre
transcendant. Vous pouvez consulter Wikipedia pour en apprendre plus.


\subsection{Définition de l'exponentielle népérienne}
\label{sec:org9189d34}
\begin{definition}
L'exponentielle Népérienne est l'exponentielle de base \(e\).
\end{definition}

Cette exponentielle Népérienne est en général choisie comme représentante de
toutes les fonctions exponentielles. Une des propriété remarquable est que cette
fonction est égale à sa dérivée:
\[
(e^x)'=e^x.
\]

Son graphique vous est donné sur la page de couverture. Voici un exercice.
\begin{exercice}
Trace le graphique de la fonction \(g(x)= \dfrac{1}{e^x}\) à partir du graphique
de l'exponentielle népérienne.

\begin{center}
\includestandalone[width=0.8\linewidth]{figures/fig23}
\end{center}
\end{exercice}

\newpage
\section{Équations exponentielles simples}
\label{sec:org09a1769}
Une équation exponentielle est une équation où l’inconnue se trouve dans un (ou
plusieurs) exposant(s).

Pour résoudre ces équations, on veillera toujours à exprimer les exponentielles
dans la même base.
\subsection{Résolution des équations de la forme \(a^{f(x)}=a^{g(x)}\)}
\label{sec:org054e06a}
La résolution des équations de la forme  \(a^{f(x)}=a^{g(x)}\) se fait en trois
étapes.

\begin{enumerate}
\item On rémène l'équation sous la forme \(a^{f(x)}=a^{g(x)}\), si cela est
nécessaire.
\item On résout ensuite \(f(x)=g(x)\).
\item On écrit l'ensemble des solutions.
\end{enumerate}

Voici deux exemples:
\par \setlength{\columnseprule}{2 pt}
          \begin{minipage}[t]{\linewidth}
          \begin{multicols}{2}
L'équation de départ est
\[
e^{3x+1}-\dfrac{1}{e^2}=0
\]

\begin{enumerate}
\item on la récrit sous la forme

\[
   e^{3x+1}=e^{-2}
   \]

\item on résout l'équation \(3x+1=-2\):

on a \(x=-1\).

\item l'ensemble des solutions est \(S=\{-1\}\).
\end{enumerate}
L'équation de départ est
\[
\left(\dfrac{5}{3}\right)^{2x-2}=\dfrac{3}{5}
\]

\begin{enumerate}
\item on la récrit sous la forme

\[
   \left(\dfrac{5}{3}\right)^{2x-2}=\left(\dfrac{5}{3}\right)^{-1}
   \]

\item on résout l'équation \(2x-2=-1\):

on a \(x=\dfrac{1}{2}\).

\item l'ensemble des solutions est \(S=\left\{\dfrac{1}{2}\right\}\).
\end{enumerate}


\end{multicols}\end{minipage}

\vspace{0.5cm}

Pourquoi peut-on passer de la 1re à la 2e étape? Le passage est autorisé par le
fait que les fonctions exponentielles sont \dotfill


\subsection{Exercices}
\label{sec:org77164a8}
\begin{exercice}
Résous les équations suivantes.

\par \setlength{\columnseprule}{0 pt}
          \begin{minipage}[t]{\linewidth}
          \begin{multicols}{3}

\begin{enumerate}
\item \(5^x=1\)
\item \(3^{x-3}=9\)
\item \(10^{x+2}=0,0001\)
\item \(2^x=\sqrt[3]{4}\)
\item \(\left(\dfrac{2}{3}\right)^x=\dfrac{3}{2}\)
\item \(e^{2x}=e\)
\item \(e^{2x^2}\cdot e^{-3x}=1\)
\item \(e^{4x^2}=\dfrac{1}{e^3}\)
\item \(3^{2x-3}=\dfrac{1}{\sqrt{3}}\)
\item \(\sqrt[5]{3}=\dfrac{1}{3^{2x+1}}\)
\end{enumerate}



\end{multicols}\end{minipage}
\end{exercice}

\begin{exercice}
Résous l'équation suivante: \(12\cdot e^{x}+5=41\). Que constates-tu?
\end{exercice}

\chapter{Fonctions logarithmiques}
\label{sec:org8bb23e6}
\section{Introduction}
\label{sec:org27046aa}
En clôturant le chapitre précédent, tu as constaté que tu manquais d'outils pour
résoudre l'équation \(12\cdot e^{x}+5=41\). En effet, en manipulant cette
équation, tu as été bloqué pour résoudre
\[
e^x=3.
\]

Un des objectifs de ce chapitre est de développer un outil pour résoudre cette
équation.

Commençons par essayer de la résoudre graphiquement, sur base du graphique de
\(e^x\).


\begin{center}
\includestandalone[width=0.6\linewidth]{figures/fig23}
\end{center}

Tu observes graphiquement que la solution de l'équation \(e^x=3\) vaut
approximativement \(x\simeq\jdot{2cm}\).

Nous allons résoudre graphiquement d'autres équations, du type \(e^x=a\), où \(a\)
est un paramètre qui prendra différentes valeurs. Complète le tableau suivant:


\begin{center}
\begin{tabular}{|p{3cm}|p{5cm}|}
\hline
\(a\) & solution de l'équation \(e^x=a\)\\[0pt]
\hline
-2 & \\[0pt]
\hline
-1 & \\[0pt]
\hline
0 & \\[0pt]
\hline
\(0,1\) & \\[0pt]
\hline
\(0,5\) & \\[0pt]
\hline
\(0,8\) & \\[0pt]
\hline
1 & \\[0pt]
\hline
2 & \\[0pt]
\hline
3 & \\[0pt]
\hline
4 & \\[0pt]
\hline
5 & \\[0pt]
\hline
6 & \\[0pt]
\hline
7 & \\[0pt]
\hline
8 & \\[0pt]
\hline
9 & \\[0pt]
\hline
10 & \\[0pt]
\hline
\end{tabular}
\end{center}

Maintenant, représente graphiquement les solutions de l'équation en fonction du
paramêtre \(a\). Relie ensuite les points par une courbe.

\begin{center}
\includestandalone[width=0.9\linewidth]{figures/fig24}
\end{center}

Tu viens de construire ta première fonction logarithmique! Cette nouvelle
fonction est notée \(\ln(x)\).

Voici le graphique complet de \(\ln(x)\) ainsi que celui de \(e^x\).

\begin{center}
\includestandalone[width=0.9\linewidth]{figures/fig25}
\end{center}

Tu constates que ces deux graphiques sont symétriques par rapport à la
bissectrice du repère: ces deux fonctions sont \textbf{réciproques} l'une de l'autre,
tout comme

\begin{enumerate}
\item \(x^2\) et \(\sqrt{x}\) sont réciproques l'une de l'autre et

\item \(x^3\) et \(\sqrt[3]{x}\) sont réciproques l'une de l'autre
\end{enumerate}

Nous allons exploiter cette réciprocité pour étudier les fonctions logarithmiques.

À l'aide de la calculatrice, complète les égalités suivantes:

\par \setlength{\columnseprule}{0 pt}
          \begin{minipage}[t]{\linewidth}
          \begin{multicols}{2}
\(\ln(e^5)=\jdot{2cm}\)

\(\ln(e^{-10})=\jdot{2cm}\)

\(\ln(e^{0,8})=\jdot{2cm}\)

\(\ln(e^{0})=\jdot{2cm}\)

\(e^{\ln(10)}=\jdot{2cm}\)

\(e^{\ln(0.01)}=\jdot{2cm}\)

\(e^{\ln(999)}=\jdot{2cm}\)

\(e^{\ln(-4)}=\jdot{2cm}\)


\end{multicols}\end{minipage}


Ainsi, la fonction \(\ln\) peut être définie de la manière suivante, en faisant
une traduction entre \(\ln(x)\) et \(e^x\): \(\ln(x)=y\) si et seulement si \(e^y=x\).


Tu viens d'observer que cette traduction est correcte en complétant les égalité
précédentes.

Le logarithme népérien, \(\ln\), peut nous aider à résoudre l'équation \(e^x=3\):

\vspace{3cm}

\section{Définition du logarithme de base \(a\)}
\label{sec:org8a260f8}

\begin{definition}
Soit \(a\in\IR^{>0}\backslash\{1\}\). Le logarithme de base \(a\), noté \(\log_a\),
est la fonction définie par la relation
\[
\log_a(x)=y \text{ si et seulement si } a^y=x.
\]
Le domaine de la fonction \(\log_a\) est \(\IR^{>0}\).
\end{definition}

\begin{exemple}
Sans utiliser la calculatrice, complète les égalités suivantes.
\par \setlength{\columnseprule}{0 pt}
          \begin{minipage}[t]{\linewidth}
          \begin{multicols}{2}

\(a=2\)

\(\log_2(4)=\jdot{2cm}\) car \(2^{\ldots}=\ldots\)

\(\log_2(\sqrt{2})=\jdot{2cm}\) car \(2^{\ldots}=\ldots\)

\(\log_2(1/2)=\jdot{2cm}\) car \(2^{\ldots}=\ldots\)

\(\log_2(2)=\jdot{2cm}\) car \(2^{\ldots}=\ldots\)

\(\log_2(-2)=\jdot{2cm}\) car \(2^{\ldots}=\ldots\)

\(\log_2(0)=\jdot{2cm}\) car \(2^{\ldots}=\ldots\)

\(a=5\)

\(\log_5(25)=\jdot{2cm}\) car \(5^{\ldots}=\ldots\)

\(\log_5(0,2)=\jdot{2cm}\) car \(5^{\ldots}=\ldots\)

\(\log_5(125)=\jdot{2cm}\) car \(5^{\ldots}=\ldots\)

\(\log_5(-125)=\jdot{2cm}\) car \(5^{\ldots}=\ldots\)

\(\log_5(\sqrt[3]{5})=\jdot{2cm}\) car \(5^{\ldots}=\ldots\)

\(\log_5(1)=\jdot{2cm}\) car \(5^{\ldots}=\ldots\)



\end{multicols}\end{minipage}
\end{exemple}

On peut expliciter la relation de réciprocité entre \(\log_a\) et \(a^x\) de la
manière suivante:

\begin{propriete}
Soit \(a\in\IR^{>0}\backslash\{1\}\). Alors
\begin{enumerate}
\item quel que soit \(x\in\IR\), \(\log_a(a^x)=x\);
\item quel que soit \(x\in\IR^{>0}\), \(a^{\log_a(x)}=x\).
\end{enumerate}
\end{propriete}

C'est grâce à cette réécriture de la définition qu'on va être capables de
démontrer les propriétés algébriques des logarithmes à partir des propriétés
algébriques des exponentielles.

\begin{exemple}
Utilise les touches \texttt{ln} et \texttt{log} de ta calculatrice pour calculer
les expressions suivantes:

\par \setlength{\columnseprule}{0 pt}
          \begin{minipage}[t]{\linewidth}
          \begin{multicols}{2}

\(\ln(e^4)=\jdot{2cm}\)

\(\ln(e^{-2})=\jdot{2cm}\)

\(\ln(e^{0,5})=\jdot{2cm}\)

\(\ln(e^{0})=\jdot{2cm}\)

\(e^{\ln(10)}=\jdot{2cm}\)

\(e^{\ln(0)}=\jdot{2cm}\)

\(e^{\ln(-1)}=\jdot{2cm}\)

\(\log(10^4)=\jdot{2cm}\)

\(\log(10^{-2})=\jdot{2cm}\)

\(\log(10^{0,5})=\jdot{2cm}\)

\(\log(10^{0})=\jdot{2cm}\)

\(10^{\log(10)}=\jdot{2cm}\)

\(10^{\log(\pi)}=\jdot{2cm}\)

\(10^{\log(1)}=\jdot{2cm}\)


\end{multicols}\end{minipage}
\end{exemple}
\subsection{Deux logarithmes particuliers: \(\ln\) et \(\log_{10}\)}
\label{sec:org1b21906}

Tu as vu que ta calculatrice a deux touches particulières: \texttt{ln} et
\texttt{log}. Ce sont respectivement les touches pour les logarithmes de base
\(e\) et \(10\): \(\ln=\log_e\) et \(\log=\log_{10}\).

Voici les graphiques de ces deux fonctions, et les exponentielles associées.


\begin{center}
\includestandalone[width=0.9\linewidth]{figures/fig26}
\end{center}

\subsection{Exercices}
\label{sec:org98bb4aa}

\begin{exercice}
Sans utiliser la calculatrice, complète les égalités suivantes en faisant une
\textbf{traduction} vers les fonctions exponentielles.

Voici un exemple: \(\log_2(64)=6\) car \(2^6=64\).

\begin{enumerate}
\item \(\log_{2}(1/8)=\jdot{2cm}\) car \(\jdot{3cm}\)

\item \(\log_{0,5}(2^5)=\jdot{2cm}\) car \(\jdot{3cm}\)

\item \(\log_{3}(0)=\jdot{2cm}\) car \(\jdot{3cm}\)

\item \(\log(\sqrt{10^3})=\jdot{2cm}\) car \(\jdot{3cm}\)

\item \(\log_{9}(3)=\jdot{2cm}\) car \(\jdot{3cm}\)

\item \(\log(0,0001)=\jdot{2cm}\) car \(\jdot{3cm}\)

\item \(\ln(1/e^3)=\jdot{2cm}\) car \(\jdot{3cm}\)

\item \(\ln(e)=\jdot{2cm}\) car \(\jdot{3cm}\)

\item \(\log(-0,01)=\jdot{2cm}\) car \(\jdot{3cm}\)

\item \(\log(\sqrt{0,001})=\jdot{2cm}\) car \(\jdot{3cm}\)
\end{enumerate}
\end{exercice}

\begin{exercice}
En utilisant la calculatrice, calcule les expressions suivantes à \(10^{-4}\) près.

\par \setlength{\columnseprule}{0 pt}
          \begin{minipage}[t]{\linewidth}
          \begin{multicols}{2}
\begin{enumerate}
\item \(\ln(2)=\jdot{3cm}\)

\item \(\ln(10)=\jdot{3cm}\)

\item \(\log(e)=\jdot{3cm}\)

\item \(\log(11)=\jdot{3cm}\)
\end{enumerate}


\end{multicols}\end{minipage}
\end{exercice}


\section{Propriétés graphiques}
\label{sec:orgff3480e}
Voici deux graphiques représentatifs des fonctions logarithmiques.
\begin{center}
\includestandalone[width=\linewidth]{figures/fig27}
\end{center}
\begin{propriete}
Soit \(a\in\IR^{>0}\backslash\{1\}\). On note \(f\) le logarithme de
base \(a\). Alors,

\par \setlength{\columnseprule}{2 pt}
          \begin{minipage}[t]{\linewidth}
          \begin{multicols}{2}
\begin{itemize}
\item \(\dom{f}=\dotfill\)
\item \(\im{f}=\dotfill\)
\item Racines=\dotfill
\item o.à.o=\dotfill
\item \(G_f\cap x=\dotfill\)
\item \(G_f\cap y=\dotfill\)
\item AV\dotfill
\item AO\dotfill
\item AH\dotfill
\item Points particuliers:\dotfill
\end{itemize}



\end{multicols}\end{minipage}

De plus,
\par \setlength{\columnseprule}{2 pt}
          \begin{minipage}[t]{\linewidth}
          \begin{multicols}{2}
\uline{\textbf{si \(a>1\)}}
\begin{itemize}
\item Concavité \dotfill
\item \(f\) est \dotfill
\item \(\lim\limits_{x\to 0}f(x)=\dotfill\)
\item \(\lim\limits_{x\to\infty}f(x)=\dotfill\)
\end{itemize}
\columnbreak
 \uline{\textbf{si \(0<a<1\)}}
\begin{itemize}
\item Concavité \dotfill
\item \(f\) est \dotfill
\item \(\lim\limits_{x\to 0}f(x)=\dotfill\)
\item \(\lim\limits_{x\to\infty}f(x)=\dotfill\)
\end{itemize}


\end{multicols}\end{minipage}
\end{propriete}
Vous pouvez de plus remarqué une symétrie des graphiques ci-dessus.
\begin{propriete}
Soit \(a\in\IR^{>0}\backslash\{1\}\). Alors le graphique de
\(f(x)=\log_{1/a}(x)\) est obtenu à partir du graphique de \(g(x)=\log_a(x)\)
par une symétrie d'axe \(x\).
\end{propriete}
\newpage
\subsection{Exercices}
\label{sec:org4b81a98}
\begin{exercice}
Donne l'expression analytiques des fonctions logarithmiques représentées
ci-dessous.
\begin{center}
\includestandalone[width=\linewidth]{figures/fig28}
\end{center}
\end{exercice}

\begin{exercice}
Associe chaque expression analytique au graphique correspondant.
\begin{center}
\begin{tabular}{|l|l|}
\hline
Expression analytique & Graphique \\ \hline
$\log(x)$             &           \\ \hline
$4^x$                 &           \\ \hline
$1/e^x$               &           \\ \hline
$\log_5(x)$           &           \\ \hline
$\log_{0,5}(x)$       &           \\ \hline
\end{tabular}
\end{center}

\begin{center}
\includestandalone[width=0.95\linewidth]{figures/fig29}
\end{center}
\end{exercice}

\begin{exercice}
Soit \(a\in\IR^{>0}\backslash\{1\}\). Relie chaque élément de la colonne de gauche
à l'information correspondante dans la colonne de droite.

\par \setlength{\columnseprule}{0 pt}
          \begin{minipage}[t]{\linewidth}
          \begin{multicols}{2}

\begin{itemize}
\item Les graphiques de \(a^x\) et \(\dfrac{1}{a^x}\)

\item Les graphiques de \(a^x\) et \(\log_a(x)\)

\item Les graphiques de \(\dfrac{1}{a^x}\) et \(\log_{\frac{1}{a}}(x)\)

\item Les graphiques de \(\log_a(x)\) et \(\log_{\frac{1}{a}}(x)\)
\end{itemize}

\columnbreak

\begin{itemize}
\item sont symétriques par rapport à l'axe \(x\)

\item sont symétriques par rapport à la droite \(y=x\)

\item sont symétriques par rapport à l'axe \(y\)
\end{itemize}



\end{multicols}\end{minipage}
\end{exercice}

Sur base de l'exercice précédent, résous les exercices suivants.

\begin{exercice}
À partir du graphique de \(f(x)=e^x\) (représentée en pointillés), donne
l'expression analytique des autres fonctions.

\begin{center}
\includestandalone[width=\linewidth]{figures/fig30}
\end{center}
\end{exercice}

\begin{exercice}
À partir du graphique de \(f(x)=\log_4(x)\) (représentée en pointillés), donne l'expression analytique des autres fonctions.
\begin{center}
\includestandalone[width=\linewidth]{figures/fig31}
\end{center}
\end{exercice}

\begin{exercice}
À partir du graphique de \(f(x)=\left(\dfrac{4}{5}\right)^x\)(représentée en pointillés) , donne l'expression
analytique de la fonction \(g(x)\) (représentée avec le trait le plus gras). Ensuite, donne l'expression analytique de la
fonction \(h(x)\) à partir de la fonction \(g(x)\).

\begin{center}
\includestandalone[width=\linewidth]{figures/fig32}
\end{center}
\end{exercice}

\begin{exercice}
Parmi les trois graphiques ci-dessous, trouve celui de la fonction
\(f(x)=\log_{0,4}(x)\). Donne ensuite l'expression analytique des autres
fonctions.
\begin{center}
\includestandalone[width=0.7\linewidth]{figures/fig33}
\end{center}
\end{exercice}
\section{Propriétés algébriques}
\label{sec:org3a26d3e}

Les propriétés algébriques des logarithmes seront déduites des propriétés
algébriques des exponentielles grâce à la relation de réciprocité entre ces deux
fonctions.

La stratégie des démonstrations des propriétés sera donc toujours la même:

\begin{boite}
traduire la propriété algébrique du logarithme en une propriété algébrique de
l'exponentielle correspondante, grâce à la définition:
\(\log_a(x)=y\Leftrightarrow x=a^y\).
\end{boite}

Voici, pour rappel, les propriétés élémentaires déjà rencontrées:

\begin{propriete}
Soit \(a\in\IR^{>0}\backslash\{1\}\). Alors:
\begin{enumerate}
\item \(\log_a(a^x)=x\) (\(x\in\IR\)) et \(a^{\log_a(x)=x}\) (\(x\in\IR^{>0}\))
\item \(\log_a(1)=0\)  (traduction: \(a^0=1\))
\item \(\log_a(a)=1\)  (traduction: \(a^1=a\))
\end{enumerate}
\end{propriete}

\subsection{Logarithme d'un produit}
\label{sec:orga3bcfb6}

\begin{propriete}
Soit \(a\in\IR^{>0}\backslash\{1\}\). Soit \(x,y\in\IR^{>0}\). Alors
\[
\log_a(xy)=\log_a(x)+\log_a(y).
\]
\end{propriete}
Ainsi le logarithme transforme un produit en une somme. La traduction
correspondante pour les exponentielle est qu'une exponentielle transforme une
somme en un produit!

\begin{proof}
On commence par traduire la propriété:
\[
\log_a(xy)=\log_a(x)+\log_a(y)\text{ devient }
xy=a^{\log_a(x)+\log_a(y)}.
\]

Puisque l'exponentielle transforme une somme en un produit:
\begin{align*}
a^{\log_a(x)+\log_a(y)} &=\jdot{5cm}\\
&=\jdot{5cm}
\end{align*}

Donc l'égalité \(xy=a^{\log_a(x)+\log_a(y)}\) est vraie, et par
conséquent  \(\log_a(xy)=\log_a(x)+\log_a(y)\).
\end{proof}

\begin{exemple}
Voici quelques exemples:

\begin{enumerate}
\item \(\log(1000\cdot 0,001)\)
\vspace{2cm}
\item \(\log_2\left(8\cdot\dfrac{1}{4}\right)\)
\vspace{2cm}
\end{enumerate}
\end{exemple}
\subsection{Logarithme d'un quotient}
\label{sec:org8024308}
\begin{propriete}
Soit \(a\in\IR^{>0}\backslash\{1\}\). Soit \(x,y\in\IR^{>0}\). Alors
\[
\log_a\left(\dfrac{x}{y}\right)=\log_a(x)-\log_a(y).
\]
\end{propriete}
Ainsi le logarithme transforme un quotient en une différence. La traduction
correspondante pour les exponentielle est qu'une exponentielle transforme une
différence en un quotient!

\begin{proof}
On commence par traduire la propriété:
\[
\log_a\left(\dfrac{x}{y}\right)=\log_a(x)-\log_a(y)\text{ devient }
\jdot{5cm}
\]
Puisque l'exponentielle transforme une différence en un quotient:
\begin{align*}
a^{\jdot{5cm}} &=\jdot{5cm}\\
&=\jdot{5cm}
\end{align*}

Donc l'égalité \(\jdot{10cm}\) est vraie, et par
conséquent  \(\log_a\left(\dfrac{x}{y}\right)=\log_a(x)-\log_a(y)\).
\end{proof}

\begin{exemple}
Voici quelques exemples:

\begin{enumerate}
\item \(\log_5\left(\dfrac{125}{5}\right)\)
\vspace{2cm}
\item \(\log_2\left(\dfrac{16}{4}\right)\)
\vspace{2cm}
\end{enumerate}
\end{exemple}
\subsection{Logarithme d'une puissance}
\label{sec:org8b88e1f}
\begin{propriete}
Soit \(a\in\IR^{>0}\backslash\{1\}\). Soit \(x\in\IR^{>0}\) et \(p\in\IR\). Alors
\[
\log_a(x^p)=p\log_a(x).
\]
\end{propriete}

\begin{proof}
On commence par traduire la propriété:
\[
\log_a(x^p)=p\log_a(x)\text{ devient }
\jdot{5cm}
\]
Puisque la puissance d'une exponentielle est l'exponentielle d'un produit (
   \(\left(a^{m}\right)^p=\jdot{2cm}\) ):
\begin{align*}
a^{\jdot{5cm}} &=\jdot{5cm}\\
&=\jdot{5cm}
\end{align*}

Donc l'égalité \(\jdot{10cm}\) est vraie, et par
conséquent  \(\log_a(x^p)=p\log_a(x)\).
\end{proof}

\begin{exemple}
Voici quelques exemples:

\begin{enumerate}
\item \(\log_5\left(125\right)\)
\vspace{2cm}
\item \(\log_2\left(16\right)\)
\vspace{2cm}
\end{enumerate}
\end{exemple}
\subsection{Changement de base}
\label{sec:org56df2ac}
Une calculatrice dispose en général de deux touche logarithmiques: \texttt{ln}
et \texttt{log}. Comment peut-on alors utiliser la calculatrice pour calculer
des logarithmes dans d'autres bases? On utilise une formule qui permet de
traduire un logarithme dans une autre base.
\begin{propriete}
Soit \(a,b\in\IR^{>0}\backslash\{1\}\). Soit \(x\in\IR^{>0}\). Alors
\[
\log_a(x)=\dfrac{\log_b(x)}{\log_b(a)}.
\]
En particulier, tout logarithme se traduit en un logarithme de base \(e\) ou \(10\)
de la manière suivante:
\[
\log_a(x)=\dfrac{\ln(x)}{\ln(a)}\text{ ou }\log_a(x)=\dfrac{\log(x)}{\log(a)}.
\]
\end{propriete}

C'est cette propriété qui permet d'utiliser la calculatrice pour calculer des
logarithmes dans des bases différentes de \(e\) et \(10\).

\begin{exemple}
Calcule à l'aide de la calculatrice les logarithmes suivants:
\begin{enumerate}
\item \(\log_3(2)\)
\vspace{2cm}

\item \(\log_2(125)\)
\vspace{2cm}

\item \(\log_{0,5}(54)\)
\vspace{2cm}
\end{enumerate}
\end{exemple}
\subsection{Exercices}
\label{sec:org9f72c6b}
\begin{boite}
\par \setlength{\columnseprule}{0 pt}
          \begin{minipage}[t]{\linewidth}
          \begin{multicols}{3}

\(\log_a(1)=0\) et \(\log_a(a)=1\)

\(a^{\log_a(x)}=x\) et \(\log_a(a^x)=x\)

\(\log_a(xy)=\log_a(x)+\log_a(y)\)

\(\log_a\left(\dfrac{x}{y}\right)=\log_a(x)-\log_a(y)\)

\(\log_a(x^p)=p\log_a(x)\)

\(\log_a(x)=\dfrac{\log_b(x)}{\log_b(a)}\)


\end{multicols}\end{minipage}
\end{boite}

\begin{exercice}
Calcule sans calculatrice en utilisant les propriétés élémentaires :
\par \setlength{\columnseprule}{0 pt}
          \begin{minipage}[t]{\linewidth}
          \begin{multicols}{3}
\begin{enumerate}
\item \(\log_2(3)\)
\item \(\log(10^7)\)
\item \(\log_8(1)\)
\item \(\log_3(3)\)
\item \(\ln(1)\)
\item \(\log_5(5^3)\)
\item \(\ln(e^5)\)
\item \(\log(10^{-5})\)
\item \(\log\left(\dfrac{1}{10^3}\right)\)
\end{enumerate}


\end{multicols}\end{minipage}
\end{exercice}

\begin{exercice}
Utilise les propriétés et transforme les expressions en les développant en
somme (et/ou différence) de multiples de \(\ln(a)\), \(\ln(b)\), \(\ln(c)\),
\(\log(x)\), \(\log(y)\) et \(\log(z)\):

\par \setlength{\columnseprule}{0 pt}
          \begin{minipage}[t]{\linewidth}
          \begin{multicols}{3}
\begin{enumerate}
\item \(\ln(ab^2c)\)
\item \(\ln\left(\dfrac{ab}{c}\right)\)
\item \(\ln(\sqrt{ab^2})\)
\item \(\log(10xy)\)
\item \(\log(x^2y^3)\)
\item \(\log(\sqrt[3]{10x})\)
\end{enumerate}


\end{multicols}\end{minipage}
\end{exercice}

\begin{exercice}
Ecris les expressions sous la forme d’un seul logarithme:
\par \setlength{\columnseprule}{0 pt}
          \begin{minipage}[t]{\linewidth}
          \begin{multicols}{2}
\begin{enumerate}
\item \(\log(3b)+\log(5a)\)

\item \(\ln(6c)-\ln(b)\)

\item \(\log(3x)+\log(y)-\log(2x)\)

\item \(\log(a)-2\log(b)+3\log(c)\)

\item \(\ln(3x)+2\ln(x)-\dfrac{1}{2}\ln(y)\)

\item \(\log(4x)+\log(x+1)-\log(y-6)\)
\end{enumerate}



\end{multicols}\end{minipage}
\end{exercice}

\begin{exercice}
Écris le plus simplement possible les expressions suivantes:
\par \setlength{\columnseprule}{0 pt}
          \begin{minipage}[t]{\linewidth}
          \begin{multicols}{2}
\begin{enumerate}
\item \(\log(\sqrt{10^3})\)
\item \(\ln\left(\dfrac{1}{e^3}\right)\)
\item \(\ln(\sqrt{e})\)
\item \(\ln(e^2)-\ln(e^{-2})\)
\end{enumerate}


\end{multicols}\end{minipage}
\end{exercice}

\begin{exercice}
Utilise la formule de changement de base pour calculer les logarithmes suivants.
\par \setlength{\columnseprule}{0 pt}
          \begin{minipage}[t]{\linewidth}
          \begin{multicols}{2}
\begin{enumerate}
\item \(\log_9(100)\)

\item \(\log_5(e)\)
\end{enumerate}


\end{multicols}\end{minipage}
\end{exercice}

\begin{exercice}
Sachant que \(\log(x)=2,4\), calcule:
\par \setlength{\columnseprule}{0 pt}
          \begin{minipage}[t]{\linewidth}
          \begin{multicols}{2}
\begin{enumerate}
\item \(\log(x^2)\)
\item \(\log\left(\dfrac{1}{x}\right)\)
\item \(\log(1000x)\)
\item \(\log(\sqrt{10x})\)
\end{enumerate}


\end{multicols}\end{minipage}
\end{exercice}

\section{Équations logarithmiques}
\label{sec:org15f7913}

Une équation logarithmique est une équation pour laquelle l'inconnue est un
argument d'une fonction logarithme.

Nous allons développer une technique pour résoudre les équations logarithmiques.

La démarche est assez similaire à la résolution d'équations exponentielles. La
différence majeure est que pour les équations logarithmiques, il \textbf{faut poser des
conditions d'existence}, étant donné que le domaine d'une fonction logarithmique
est \(\IR^{>0}\).

Voyons comment résoudre l'équation \(\log(2)+\log(x)=\log(x-2)\).

\begin{enumerate}
\item On pose les conditions d'existence:
\vspace{3cm}
\item On résout l'équation en utilisant les propriétés des logarithmes.
\vspace{4cm}

L'astuce principale pour résoudre les équations logarithmiques est donc:
\vspace{2cm}
\item On confronte les solutions de la 2e étape aux conditions d'existence de la
première étape.
\vspace{2cm}
\item On conclut:
\end{enumerate}

\subsection{Exercices}
\label{sec:org7f82d04}
\begin{exercice}
Résous les équations suivantes.

\par \setlength{\columnseprule}{0 pt}
          \begin{minipage}[t]{\linewidth}
          \begin{multicols}{2}
\begin{enumerate}
\item \(\log(x-2)=4\)
\item \(\log(x+3)=\log(x)-\log(12)\)
\item \(\log(4x-3)-\log(2x-4)=\log(25)\)
\item \(\log(4x-3)+\log(4)=\log(2x-4)\)
\item \(2\log(x)=\log(x+2)\)
\item \(2\ln(3x+1)=0\)
\item \(\log(3x-2)+\log(8x)=\log(8)\)
\item \(\log(4x)=-2\)
\end{enumerate}


\end{multicols}\end{minipage}
\end{exercice}
\section{Résolution d'équations exponentielles de la forme \(a^{f(x)}=b\)}
\label{sec:orga04e9a6}

Nous avons vu au chapitre précédent qu'il était impossible de résoudre
l'équation \(e^x=3\) sans un nouvel outil. Nous allons voir comment les
logarithmes permettent de résoudre ce genre d'équations.

Nous allons nous concentrer sur les équations qui se ramènent par manipulations
algébriques à des équations du type

\[
a^{f(x)}=b, \text{ où } a\in\IR^{>0}\backslash\{1\} \text{ et } b\in\IR^{>0}.
\]

Lorsque vous arrivez à une équation du type \(a^{f(x)}=b\), vous pouvez appliquer
le logarithme de base \(a\) à chacun des membres de l'équations:

\[
a^{f(x)}=b\text{ si et seulement si } \log_a(a^{f(x)})=\log_a(b).
\]
Ainsi, grâce à la relation de réciprocité, on obtient

\[
\log_a(a^{f(x)})=\log_a(b) \text{ si et seulement si } f(x)=\log_a(b).
\]

Il suffit ensuite de résoudre l'équation \(f(x)=\log_a(b)\).

Voici quatre exemples

\begin{exemple}
Résous les équations suivantes.

a. \(e^{1-3x}=3\)

\vspace{3cm}

b. \(2\cdot 10^{5x}=12\)

\vspace{3cm}

c. \(5=2^{x+5}\)

\vspace{3cm}

d. \(2^{x-1}=-8\)
   \vspace{2cm}
\end{exemple}

\begin{exercice}
Résous les équations suivantes.

\par \setlength{\columnseprule}{0 pt}
          \begin{minipage}[t]{\linewidth}
          \begin{multicols}{3}

\begin{enumerate}
\item \(3^{2x}=7\)

\item \(\dfrac{1}{3} 2^{x+2}=7\)

\item \(e^{\dfrac{x}{8}}=3^2\)

\item \(2^{x+1}=3\)

\item \(6e^{2x}-1=0\)

\item \(5^{3x-9}=8\)
\end{enumerate}



\end{multicols}\end{minipage}
\end{exercice}
\section{Modélisations à l'aide de fonctions logarithmiques}
\label{sec:orge0ec71f}
\begin{exercice}
(Population mondiale) Selon une estimation de l’ONU (Organisation des Nations Unies), la population
mondiale était en 2005 de 6,5 milliards d’habitants et est depuis en progression de
1,2 \% par an.
Détermine l’année où la population sera doublée, si cette progression se déroule
toujours à ce rythme.
\end{exercice}

\begin{exercice}
(Placement) Une personne effectue un placement de 4000 euros à intérêts
composés, au taux annuel de 5 \%. Cela signifie que la somme
d’argent va augmenter de 5\% tous les ans.
Au bout de combien d’années son capital sera-t-il égal à 8000\texteuro{} ?
\end{exercice}

\begin{exercice}
(Médicament) On injecte à une personne 250 mg d’un médicament dans le sang.
La dose de médicament contenue dans le sang en fonction du temps est modélisé par
la fonction :

\[
d(t)=d_0\cdot 10^{-0,075t}
\]
où \(d_0\) est la dose introduite dans le sang et \(t\) est le temps est mesuré en heure.
Si la dose de médicament contenue dans le sang est inférieure à 100 mg, on estime
que celui-ci n’a plus d’effet. Il est temps de réinjecter une dose. Après combien de
temps faudra-t-il le faire?
\end{exercice}

\section{Question de dépassement}
\label{sec:org533a9a1}

\begin{question}
Pourquoi n'avons-nous pas parlé de multiples de fonctions logarithmiques?
\end{question}
\chapter{Dérivées et limites des fonctions exponentielles et logarithmes}
\label{sec:org3b19d65}
\uline{Attention!} Ce chapitre se fera en partie par prise de notes pour la théorie!

\section{Calculs de dérivées}
\label{sec:org761211b}

\begin{exercice}
Calcule les dérivées suivantes.
\par \setlength{\columnseprule}{0 pt}
          \begin{minipage}[t]{\linewidth}
          \begin{multicols}{5}
a) \(3^x\)

b) \(10^x\)

c) \(e^{-x}\)

d) \(\dfrac{1}{e^{4x}}\)

e) \(e^{x^2}\)

f) \(e^{2x-2}\)

g) \(e^{1/x}\)

h) \(3^{4x-3}\)

i) \(2^{1/x}\)

j) \(4^{x/3}\)

k) \(2e^x\)

l) \(\dfrac{1}{e^{x}}+e^{x}\)

m) \(\dfrac{1}{e^{3x}}+e^{4x}\)

n) \(xe^x\)

o) \(x^2e^{4x}\)

p) \(\dfrac{e^x}{x}\)

q) \(\dfrac{e^x}{x^3}\)

r) \(\dfrac{2}{e^{2x}}\)


\end{multicols}\end{minipage}
\end{exercice}

\begin{exercice}
Calcule les dérivées suivantes.

\par \setlength{\columnseprule}{0 pt}
          \begin{minipage}[t]{\linewidth}
          \begin{multicols}{5}

a) \(\log_2(x)\)

b) \(\ln(3x)\)

c) \(\ln(x^3)\)

d) \(\ln(1/x)\)

e) \(\ln(\sqrt{x})\)

f) \(\ln(3x-5)\)

g) \(\ln(x)-x\)

h) \(x^2\ln(x)\)

i) \(x\ln(x)-x\)

j) \(\dfrac{\ln(x)}{x}\)

k) \(x^4\ln(x^2-1)\)


\end{multicols}\end{minipage}
\end{exercice}

\begin{exercice}
Calcule le domaine et la dérivée des fonctions suivantes.

\par \setlength{\columnseprule}{0 pt}
          \begin{minipage}[t]{\linewidth}
          \begin{multicols}{2}
\(f(x)=\dfrac{x}{\ln(x)}\)

\(g(x)=\ln(x^2-5x+6)\)


\end{multicols}\end{minipage}
\end{exercice}

\section{Limites de base}
\label{sec:org63e91e4}
\begin{exercice}
Calcule les limites suivantes après avoir esquissé rapidement le graphique de la
fonction.
\par \setlength{\columnseprule}{0 pt}
          \begin{minipage}[t]{\linewidth}
          \begin{multicols}{3}

a) \(\lim_{\limits{x\to \pinf }} 4^x =\)

b) \(\lim_{\limits{x\to \minf}}\left(\dfrac{1}{3}\right)^{2x} =\)

c) \(\lim_{\limits{x\to \pinf}} 2^{-x} =\)

d) \(\lim_{\limits{x\to \minf }}0,3^{x} =\)

e) \(\lim_{\limits{x\to\minf }}0,3^{x-1} =\)

f) \(\lim_{\limits{x\to \pinf}}\ln(x+1) =\)



\end{multicols}\end{minipage}
\end{exercice}

\begin{exercice}
En utilisant les règles de calcul de limites, calcule les limites suivantes.

\par \setlength{\columnseprule}{0 pt}
          \begin{minipage}[t]{\linewidth}
          \begin{multicols}{2}

a) \(\lim_{\limits{x\to \pinf }} x4^x =\)

b) \(\lim_{\limits{x\to 2}}x3^x =\)

c) \(\lim_{\limits{x\to 0}} \dfrac{0,5^x-1}{0,5^x+1} =\)

d) \(\lim_{\limits{x\to 2 }}\dfrac{x}{3^x} =\)



\end{multicols}\end{minipage}
\end{exercice}

\begin{exercice}
Calcule les limites suivantes.

\par \setlength{\columnseprule}{0 pt}
          \begin{minipage}[t]{\linewidth}
          \begin{multicols}{3}

a) \(\lim_{\limits{x\to \pinf }} \dfrac{0,5^x-1}{0,5^x+1} =\)

b) \(\lim_{\limits{x\to \minf}}\dfrac{2^x-1}{2^x+1} =\)

c) \(\lim_{\limits{x\to \minf}} \dfrac{e^x-1}{e^x+1} =\)



\end{multicols}\end{minipage}
\end{exercice}
\section{Théorème de l'Hospital}
\label{sec:org7d10cb5}
Nous n'avons pas d'outils pour calculer les limites suivantes:

$$
\lim_{\limits{x\to \pinf}}\dfrac{e^x}{2x}\text{ et }\lim_{\limits{x\to
1}}\dfrac{\ln(x)}{x-1}
$$

On ne peut pas calculer ces limites avec les techniques de base car elles
correspondent à deux cas d'indétermination. Le théorème suivant va nous
permettre de les calculer.
\begin{theoreme}
Soit \(f\) et \(g\) deux fonctions dérivables et \(a\in\IR\cup\{\pm\infty\}\). Supposons que:
\begin{enumerate}
\item \(\lim_{\limits{x\to a}}f(x)=0=\lim_{\limits{x\to a}}g(x)\) ou
\(\lim_{\limits{x\to a}}g(x)=\pm\infty\)
\item \(g'\) n'a qu'un nombre fini de racines au voisinage de \(a\)
\item \(\lim_{\limits{x\to a}}\dfrac{f'(x)}{g'(x)}\) existe.
\end{enumerate}

Alors
$$
\lim_{\limits{x\to a}}\dfrac{f(x)}{g(x)}=\lim_{\limits{x\to a}}\dfrac{f'(x)}{g'(x)}
$$
\end{theoreme}
\begin{remarque}
\begin{enumerate}
\item Toutes les hypothèses du théorèmes sont importantes. Cependant, pour les
exercices du cours, on supposera qu'elles sont vérifées.
\item Dans certains calculs de limites, il sera parfois nécessaire d'appliquer
plusieurs fois le théorème de l'Hospital.
\end{enumerate}
\end{remarque}
\begin{exemple}
\par \setlength{\columnseprule}{0 pt}
          \begin{minipage}[t]{\linewidth}
          \begin{multicols}{2}
\(\lim_{\limits{x\to \pinf}}\dfrac{e^x}{2x}\)

\(\lim_{\limits{x\to 1}}\dfrac{\ln(x)}{x-1}\)



\end{multicols}\end{minipage}
\vspace{4cm}
\end{exemple}
\begin{exercice}
Calcule les limites suivantes.
\par \setlength{\columnseprule}{0 pt}
          \begin{minipage}[t]{\linewidth}
          \begin{multicols}{2}
\begin{enumerate}
\item \(\lim_{\limits{x\to \minf}}\dfrac{e^{-3x}}{x}\)
\item \(\lim_{\limits{x\to 0}}\dfrac{e^{3^x}-3^{-x}}{3x}\)
\item \(\lim_{\limits{x\to \pinf}}\dfrac{3x}{e^x}\)
\item \(\lim_{\limits{x\to 2}}\dfrac{x-2}{\ln((x-2)^2+1)}\)
\end{enumerate}


\end{multicols}\end{minipage}
\end{exercice}
\chapter{Échelles logarithmiques}
\label{sec:org6c91c7c}
\end{document}
