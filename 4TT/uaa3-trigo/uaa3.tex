% Options for packages loaded elsewhere
\PassOptionsToPackage{unicode}{hyperref}
\PassOptionsToPackage{hyphens}{url}
%
\documentclass[
  a4paper,
  oneside]{book}

\usepackage{amsmath,amssymb}
\usepackage{iftex}
\ifPDFTeX
  \usepackage[T1]{fontenc}
  \usepackage[utf8]{inputenc}
  \usepackage{textcomp} % provide euro and other symbols
\else % if luatex or xetex
  \usepackage{unicode-math}
  \defaultfontfeatures{Scale=MatchLowercase}
  \defaultfontfeatures[\rmfamily]{Ligatures=TeX,Scale=1}
\fi
\usepackage{lmodern}
\ifPDFTeX\else  
    % xetex/luatex font selection
\fi
% Use upquote if available, for straight quotes in verbatim environments
\IfFileExists{upquote.sty}{\usepackage{upquote}}{}
\IfFileExists{microtype.sty}{% use microtype if available
  \usepackage[]{microtype}
  \UseMicrotypeSet[protrusion]{basicmath} % disable protrusion for tt fonts
}{}
\makeatletter
\@ifundefined{KOMAClassName}{% if non-KOMA class
  \IfFileExists{parskip.sty}{%
    \usepackage{parskip}
  }{% else
    \setlength{\parindent}{0pt}
    \setlength{\parskip}{6pt plus 2pt minus 1pt}}
}{% if KOMA class
  \KOMAoptions{parskip=half}}
\makeatother
\usepackage{xcolor}
\setlength{\emergencystretch}{3em} % prevent overfull lines
\setcounter{secnumdepth}{-\maxdimen} % remove section numbering
% Make \paragraph and \subparagraph free-standing


\providecommand{\tightlist}{%
  \setlength{\itemsep}{0pt}\setlength{\parskip}{0pt}}\usepackage{longtable,booktabs,array}
\usepackage{calc} % for calculating minipage widths
% Correct order of tables after \paragraph or \subparagraph
\usepackage{etoolbox}
\makeatletter
\patchcmd\longtable{\par}{\if@noskipsec\mbox{}\fi\par}{}{}
\makeatother
% Allow footnotes in longtable head/foot
\IfFileExists{footnotehyper.sty}{\usepackage{footnotehyper}}{\usepackage{footnote}}
\makesavenoteenv{longtable}
\usepackage{graphicx}
\makeatletter
\def\maxwidth{\ifdim\Gin@nat@width>\linewidth\linewidth\else\Gin@nat@width\fi}
\def\maxheight{\ifdim\Gin@nat@height>\textheight\textheight\else\Gin@nat@height\fi}
\makeatother
% Scale images if necessary, so that they will not overflow the page
% margins by default, and it is still possible to overwrite the defaults
% using explicit options in \includegraphics[width, height, ...]{}
\setkeys{Gin}{width=\maxwidth,height=\maxheight,keepaspectratio}
% Set default figure placement to htbp
\makeatletter
\def\fps@figure{htbp}
\makeatother

\usepackage{my-conf}
\makeatletter
\@ifpackageloaded{caption}{}{\usepackage{caption}}
\AtBeginDocument{%
\ifdefined\contentsname
  \renewcommand*\contentsname{Table des matières}
\else
  \newcommand\contentsname{Table des matières}
\fi
\ifdefined\listfigurename
  \renewcommand*\listfigurename{Liste des Figures}
\else
  \newcommand\listfigurename{Liste des Figures}
\fi
\ifdefined\listtablename
  \renewcommand*\listtablename{Liste des Tables}
\else
  \newcommand\listtablename{Liste des Tables}
\fi
\ifdefined\figurename
  \renewcommand*\figurename{Figure}
\else
  \newcommand\figurename{Figure}
\fi
\ifdefined\tablename
  \renewcommand*\tablename{Table}
\else
  \newcommand\tablename{Table}
\fi
}


\ifLuaTeX
\usepackage[bidi=basic]{babel}
\else
\usepackage[bidi=default]{babel}
\fi
\babelprovide[main,import]{french}
% get rid of language-specific shorthands (see #6817):
\let\LanguageShortHands\languageshorthands
\def\languageshorthands#1{}
\ifLuaTeX
  \usepackage{selnolig}  % disable illegal ligatures
\fi
\usepackage{bookmark}

\IfFileExists{xurl.sty}{\usepackage{xurl}}{} % add URL line breaks if available
\urlstyle{same} % disable monospaced font for URLs
\hypersetup{
  pdflang={fr},
  hidelinks,
  pdfcreator={LaTeX via pandoc}}


\author{}
\date{}

\begin{document}
\frontmatter


\mainmatter
\begin{titlepage}

\begin{center}

\vspace{2cm}
\Huge

\textbf{4UAA3: Trigonométrie} \vspace{4cm}

\includestandalone[width=\linewidth]{figures/fig1}

\end{center}

\end{titlepage}
\tableofcontents
\chapter{Aire d'un triangle
quelconque}\label{aire-dun-triangle-quelconque}

Dans cette UAA, nous prendrons l'habitude d'annoter les triangles de la
manière suivante:

\begin{exercice}

Soit \(ABC\) le triangle dont le côté \(b\) mesure 10cm et tel que
\(\alpha=60^\circ\) et \(\gamma=45^\circ\).

\begin{enumerate}
\def\labelenumi{\arabic{enumi}.}
\tightlist
\item
  Tracez le triangle \(ABC\) avec précision.
\item
  À l'aide du dessin, estimez l'aire du triangle \(ABC\).
\item
  Calculez l'aire du triangle \(ABC\).
\end{enumerate}

\end{exercice}

\begin{exercice}

Soit \(ABC\) le triangle dont le côté \(c\) mesure 6cm et tel que
\(\alpha=30^\circ\) et \(\beta=70^\circ\).

\begin{enumerate}
\def\labelenumi{\arabic{enumi}.}
\tightlist
\item
  Tracez le triangle \(ABC\) avec précision.
\item
  À l'aide du dessin, estimez l'aire du triangle \(ABC\).
\item
  Calculez l'aire du triangle \(ABC\).
\end{enumerate}

\end{exercice}

\begin{generalisation}
Étant donné un triangle \(ABC\), on peut calculer l'aire de ce triangle
grâce à la formule: \[
  \text{Aire}(ABC)=
\] Autrement dit: pour calculer l'aire d'un triangle, \vspace{2cm}

\end{generalisation}

\begin{observation}
Quelle est l'idée qui a permis de construire la formule du calcul
d'aire? \vspace{4cm}

\end{observation}

\chapter{Lois des sinus et cosinus}\label{lois-des-sinus-et-cosinus}

\section{Résolution de triangles}\label{ruxe9solution-de-triangles}

\section{Généralisations}\label{guxe9nuxe9ralisations}

\subsection{Loi des sinus}\label{loi-des-sinus}

\subsection{Loi des cosinus}\label{loi-des-cosinus}

\subsection{Formule de l'aire d'un triangle
quelconque}\label{formule-de-laire-dun-triangle-quelconque}

\section{Un petit problème\ldots{}}\label{un-petit-probluxe8me}

\chapter{Trigonométrie dans une triangle
quelconque}\label{trigonomuxe9trie-dans-une-triangle-quelconque}

\section{Angles orientés}\label{angles-orientuxe9s}

\section{Cercle trigonométrique}\label{cercle-trigonomuxe9trique}

\section{Relations fondamentales}\label{relations-fondamentales}


\backmatter


\end{document}
